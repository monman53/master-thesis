\chapter{実験方法}
\label{sec:model}

本研究では主にDNAゲルカプセルのパターン形成メカニズムを解明するために数値シミュレーションの実験を行った.
そのために粘弾性相分離という特殊な相分離現象のモデルを元に,
DNAゲルカプセルに対応するような拡張モデルを開発し,そのシミュレーションプログラムを実装した.
また,得られた構造を実際のDNAゲルカプセルと比較するための解析手法についても新たに考案した.
本章では拡張モデルの説明と解析手法の詳細について述べる.

\section{数理モデル}

\subsection{粘弾性相分離}
油と水の混合物は油相と液相に直ちに分離する.
このように複数成分からなる物質が同じ成分同士で分離する現象のことを相分離現象と言う.
液体における通常の相分離では少数成分が液滴を形成し安定状態に落ち着くが,
少数成分を構成する分子が,その粘弾性力などの相互作用の影響で拡散が遅い場合,
少数成分が液滴を形成する前に少数成分同士でネットワークを形成し多数成分が液滴を形成するという通常とは逆の状態が過渡期に観測される.
このような現象は粘弾性相分離と呼ばれ~\cite{tanaka2009formation},
粘性や弾性といった構成分子の動力学的な影響がネットワーク構造などのパターン形成に重要な役割を果たすと考えられている(図~\ref{fig:veps}).
本研究で注目しているDNAゲルカプセルにおけるネットワーク構造についても,
先の実験結果からY-DNAのsticky-endsによる粘弾性がそのパターン形成に寄与している可能性があり,
粘弾性相分離現象で説明ができるのではないかと考えた.

\begin{figure}
\centering
\includesvg{image/veps.svg}
\caption{
    粘弾性相分離現象の模式図.
    Tanakaらの論文~\cite{tanaka2009formation}より.
}

\label{fig:veps}
\end{figure}

\subsection{数理モデル}
今回,粘弾性相分離現象の数値解析を行うにあたりArakiらのモデル~\cite{araki2005simple}を用いた.
このモデルでは2次元空間での粘弾性相分離を仮定し,
少数成分を疎視化するラグランジュ型のモデルである.

はじめに粒子(疎視化点)は六方格子状に一様に配置され,
各粒子は隣接する6つの粒子と自然長$0$のバネで接続される.
この状態から,以下のランジュバン方程式を数値的に解いていく.


\begin{figure}
\centering
\includesvg{image/model_2d.svg}
\caption{
    Araki~\cite{araki2005simple}らのモデルの概要.
    (a) 初期状態.計算点の粒子が六方格子上に等間隔に配置され,
        隣接する6つの粒子と自然長$0$で長さ$l_0$のバネにより接続される.
        空間は周期境界条件が適用される.
    (b) 初期配置の拡大図.
    (c) シミュレーションの時間発展の様子.
        粒子の熱ゆらぎと共に,
        接続されていたバネが次第に切断されていくことで粘弾性相分離現象が再現される.
}
\label{fig:model_2d}
\end{figure}


\begin{equation}
\label{eq:main}
\zeta
\frac{d}{dt}
\vec{R}_i
=
-\frac{\partial}{\partial\vec{R}_i}
U_{tot}\{\vec{R}_i\}
+\vec{\xi}_i
.
\end{equation}

$\vec{R}_i$は粒子$i$の位置ベクトル,$\zeta$は粘性抵抗を意味する.
$\vec{\xi}_i$は以下の揺動散逸定理を満たす粒子iにおける熱ゆらぎの項である.

\begin{eqnarray}
\label{eq:langevin0}
\langle\vec{\xi}_i\rangle &=& 0, \\
\label{eq:langevin1}
\langle\vec{\xi}_i(t):\vec{\xi}_j(t')\rangle &=& 2T\zeta\delta_{ij}\delta(t-t')\bm{I}.
\end{eqnarray}

$U_{tot}$は粒子$i$の総ポテンシャルであり,
次のように2つのポテンシャルの和で表される.

\begin{eqnarray}
U_{tot}\{\vec{R}_i\}
&=
 U_{LJ}\{\vec{R}_i\}
+U_{sp}\{\vec{R}_i\}.
\end{eqnarray}

$U_{LJ}$,$U_{sp}$はそれぞれレナード・ジョーンズポテンシャルとバネによる調和ポテンシャルであり,
定義は次のとおりである.

\begin{eqnarray}
U_{LJ}\{\vec{R}_i\}
&=&
4\epsilon\sum_{j\neq i}
\left\{
\left(
\frac{\sigma}{\bigl|\vec{R}_j-\vec{R}_i\bigr|}
\right)^{12}
-
\left(
\frac{\sigma}{\bigl|\vec{R}_j-\vec{R}_i\bigr|}
\right)^{6}
\right\},
\\
U_{sp}\{\vec{R}_i\}
&=&
\frac{1}{2}
\kappa
\sideset{}{'}\sum_{j\neq i}
\bigl|\vec{R}_j-\vec{R}_i\bigr|^{2}.
\end{eqnarray}

$\epsilon$,$\sigma$,$\kappa$はそれぞれポテンシャルを特徴づける定数であり,
調和ポテンシャルにおける記号$\sideset{}{'}\sum$はバネにより接続された粒子間での和を意味する.

初期状態で接続されていたバネは,時間が経つにつれランダムに切断されていく.
その切断確率$p$は温度$T$とバネの長$l$に依存する形で定義され,
各計算ステップで全バネに対して切断するかしないかの処理がなされる.

\begin{eqnarray}
p(l)
&=&
t_0^{-1}
\exp\left(-\frac{\Delta E(l)}{T}\right)
,\\
\Delta E(l)
&=&
E_0-\frac{\kappa l^2}{2}
\end{eqnarray}

具体的には,$0$から$1$の一様乱数$r$を発生させ,$r<p(l)$ならバネを切断する.
$t_0$は数値計算の1ステップの時間で,$E_0$はバネの切れやすさを意味する定数である.
一度切断されたバネは二度と繋がることはなく,粒子間で新たにバネが接続されることもない.


\subsection{温度項の拡張}
実際のゲルカプセルの作製では温度は一定ではなく,
線形に温度を下げていくアニーリングが施される.
この状況を再現するために,温度項を次のように変更した.

\begin{eqnarray}
\label{eq:thermal}
T = T_{max}(1-\frac{t}{t_{max}})
\end{eqnarray}

$T_{max}$は初期状態の温度で,$t_{max}$はシミュレーションの終了時間である.
つまり$t=0$で$T=T_{max}$,
$t=t_{max}$で$T=0$になるような線形アニーリングを再現している.


\subsection{球面上への拡張}
カプセル状での粘弾性相分離現象をシミュレーションするために,
本研究では上記のモデルの球面上への拡張を行った.

まず,ある大きさのGeodesic Grid~\cite{Geodesic}を用意し,
その頂点に粒子を配置し辺の部分をバネとすることで初期状態とする.
この状態からランジュバン方程式を解き,移動した$R_i$を球面上に射影し直すことで常に粒子が球面上にとどまるように計算を繰り返していく.
他の処理に関しては元のモデルと同じである.

%TODO webpage の ref



\section{実装}
本モデルがラグランジュ型のモデルであり,
バネの切断などの特殊な処理が必要であることから,
汎用のシミュレーションソフトウェアでの実験は困難であると考えた.
しかし,モデルは単純であり実装は困難ではないと判断したためシミュレーションプログラムは自ら実装した.

まず,オイラー法による計算を考え,運動方程式(\ref{eq:main})を次のように離散化した.
\begin{eqnarray}
    \vec{R}_i(t+\Delta t) &=& 
    \vec{R}(t)_i
    -\frac{1}{\zeta}\frac{\partial}{\partial\vec{R}_i(t)}U(\vec{R}_i(t))\Delta t
    +\chi\sqrt{\frac{2k_B T}{\zeta}\Delta t}
    .
\end{eqnarray}

次に,各物理量をそれぞれ
$x=l_0 \tilde{x}$,
$t=t_0 \tilde{t}$,
$\zeta=\kappa t_0 \tilde{\zeta}$,
$E=\frac{1}{2}\kappa\l_0^2\tilde{E}$
でスケールし無次元化した.
以下,特に記述がなければ物理量は無次元化されたものとする.

また,元のモデルでは計算結果は頂点と辺(バネ)のみであるが,
相分離により形成された孔の解析(次節参照)を行うためにポリゴン化を行った.
ポリゴンを構成する三角形は,ある計算点と隣接する計算点で作られる三角形を全て用いた.
また,計算結果をわかりやすくするために配色を変更した.

\begin{figure}
\centering
\includesvg{image/polygon.svg}
\caption{
    計算結果のポリゴン化と配色の変更.
    (a) 変更前の計算結果のCG.
    (b) 同じ計算結果に対する変更の適用後のCG.
}
\label{fig:model_2d}
\end{figure}


プログラムはC++言語で記述し,数値計算には線形代数ライブラリEigen~\cite{eigenweb}を使用した.
データの可視化にはParaView~\cite{paraview}とPOV-Ray~\cite{povray}をそれぞれ用いた.
計算はラップトップPC上(Intel Core i5-7300U CPU)で並列実行した.


\section{解析}
シミュレーションで得られた構造を定量的に解析する必要がある.
本研究では構造の特徴として孔の大きさの分布を考えた.

得られた構造のポリゴンとその構造に対して十分小さなグリッドを持つグリッドを重ね合わせ,
三角形が重なり合わないグリッドの連続領域を深さ優先探索などのアルゴリズムにより検出し,
その領域を孔とみなして面積などの特徴量を解析に用いる手法を開発した.


\begin{figure}
\centering
\includesvg{image/xor.svg}
\caption{
    孔の検出方法.
    検出したい構造とグリッドを重ね合わせ,
    両者の全ての三角形対に対して三角形の交差判定を行うことで,
    両者が重なり合わない部分,つまり孔を近似的に抽出することができる.
    (a) シミュレーションにより得られた構造のポリゴン.
    (b) 検出に用いる十分に細かいグリッド.
    (c) 両者を重ね合わせた状態.
    (d) 構造のポリゴンと交差しないグリッド上の三角形.
}
\label{fig:xor}
\end{figure}


孔の検出で行いたい処理は,
得られた構造のポリゴンの三角形とグリッドを構成するポリゴンの三角形との交差判定を繰り返し行うことで実現できる.
三角形の交差判定に必要な計算機幾何の基本的なアルゴリズムについて説明する.

\subsection{点と直線の位置関係}
2次元平面の2点が,ある直線を堺にどちら側に存在するかを議論する方法について説明する.
平面上の2つのベクトルの外積の値は,時計回りに演算された場合に正に,半時計回りに演算された場合に負になる.
この事実を用いると,直線上の2点$q_1$,$q_2$を選び,
調べたい点$p$からの2つのベクトル$a=q_1-p$,$b=q_2-p$の外積$(a)\times(b)$の符号を調べれば良い.
例えば2点$p_1$,$p_2$が直線を堺に同じ側に存在する場合は,この外積の符号が一致する.
逆に直線を堺に分けられている場合は外積の符号が一致しない.

\subsection{線分と線分の交差判定}
点と直線の位置関係を応用して,線分と線分の交差判定を行うことができる.
つまり線分と線分の交差は4回の外積演算により判定することができる.

\begin{figure}
\centering
\includesvg{image/segment_segment.svg}
\caption{
    点と直線の位置関係をベクトルの外積を用いて議論する方法と,
    それを応用した線分と線分の交差判定.
    (a) 直線上の2点$q_1$と$q_2$をこの順番を保ったまま$p$からの2本のベクトルを引き,
        その外積の符号を考える.
        $p$を直線に対して左右に移動するとその外積の符号は変化する.
    (b) 線分と線分の交差判定は,
        線分の2つの端点がもう一方の直線に対して反対側に存在することを2線分両方について言えば良い.
        つまり4回の外積演算で線分と線分の交差判定を行うことができる.
}
\label{fig:segment_segment}
\end{figure}

\subsection{点と三角形の交差判定}
三角形の3辺に対して,ある点が全て同じ側に存在する(3つの外積の符号が一致する)場合,
その点は三角形の内部にあることが言える.
つまり点が三角形に内包されるか否かを3回の外積演算により判定することができる.

\begin{figure}
\centering
\includesvg{image/point_triangle.svg}
\caption{
    点の三角形内外判定アルゴリズム.
    点$p$から点$q_1$,$q_2$(あるいは$q_2$,$q_3$や$q_3$,$q_1$)へのベクトルの外積を調べ,
    それら3つの値の符号を見ることにより判定できる.
    (a) 点が三角形の外側にある場合は外積の符号が一致しない.
        左2つは外積の値が負になるが,一番右のケースで正になる.
    (b) 点が内側にある場合は外積の符号が全て一致する.
}
\label{fig:point_triangle}
\end{figure}

\subsection{三角形と三角形の交差判定}
三角形と三角形の交差判定は,先に説明した線分と線分の交差判定と点と三角形の交差判定を複数回行うことで実現できる.
例えば,三角形と三角形の位置関係は図\ref{fig:triangle_triangle}にあるような場合が考えられる.
図\ref{fig:triangle_triangle}(c)のような場合に関しては線分と線分の交差判定が必要になる.
%TODO 図

\begin{figure}
\centering
\includesvg{image/triangle_triangle.svg}
\caption{
    三角形と三角形の位置関係の例.
    (a) 交差しない場合.
    (b) 一方の頂点が他方に内包される場合.
    (c) 一方がもう一方に完全に内包される場合.
    (d) 一方の頂点が他方に内包されないが三角形が交差している場合.
        (a)から(c)は三角形と点の交差判定を行うだけで判定できるが,
        (d)を検出するためは線分の交差判定を行う必要がある.
}
\label{fig:triangle_triangle}
\end{figure}

\subsection{球面上の三角形の交差判定}
2次元平面では正方格子を用いて孔の面積を算出することも可能であるが,
球面上で正方格子を考えることは非常に煩雑であり,
また今回はGeodesic Gridを利用しているため,
先に説明した三角形と三角形の交差判定を用いた.

球面上の三角形の交差判定についても,2次元平面で外積を考えたように,
級の中心を原点とする3つの位置ベクトルからなる行列の行列式の符号をみることで判定することができる.

