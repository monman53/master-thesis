\chapter{結論}

自然界には様々なパターン形成現象が存在し,
我々はそれらを数理的に解析することで基礎的な研究に限らず工学的応用も行ってきた.
近年,DNA分子を用いたDNAナノテクノロジーが盛んに研究され,
新しいナノ・マイクロスケールのマテリアルやパターン形成制御法として確立されつつある.
DNA分子の利点はそのプログラマビリティにあり,
他の物質では困難な分子構造や物性などを人間が設計できる可能性を持つ.
本研究ではその中でもY-DNAと呼ばれるDNA分子について注目し,
Y-DNAが作り出す微小なゲルカプセルにみられるパターン形成メカニズムに関して研究を行った.

DNAゲルカプセルには大きく分けて,孔のようなパターンが形成される場合と,
比較的孔が少なく一様なカプセルが形成される場合の2つが存在する.
既に存在する実験結果と今回補助的に行ったIn Vitroの実験の結果などをふまえ,
カプセルにおける孔の形成が粘弾性相分離と呼ばれる現象に起因すると仮定した(第\ref{sec:dnagel}章).
私は,既存の粘弾性相分離の数理モデルを元に,DNAゲルカプセル用の新しい拡張モデルを構築し,
そのパターン形成の数値シミュレーションを行った.
また,そのパターンについての定量的な議論を行うための解析手法についても開発を行い,
両者のコンピュータプログラムの実装を行った(第\ref{sec:model}章).

数値シミュレーションの結果,
実際のDNAゲルカプセルで見られるパターン形成と非常に類似した現象を再現することに成功した.
得られた構造は,形成された孔の大きさの分布を見ることで定量的な解析を行った,
温度や粘性抵抗のパラメータを変え,それらの条件依存性について議論した結果,
先行研究の結果でえられていたパターン形成の傾向についての説明を肯定的に行うことができた(第\ref{result}章).

上記の結果は,DNAゲルカプセルに見られるパターンの形成のメカニズムが
粘弾性相分離現象によるものである可能性を結論付け,
さらにそのパターン制御の可能性についても示唆を与えた.

Y-DNA分子はsticky-endsの長さを変更するとその熱力学的特性の変化から,
Y-DNA集合体全体の物性が変化するという結果が報告されている~\cite{sato2019sequence}.
本研究で扱ったDNAカプセルでは,
温度変化やカプセルの大きさ(粘性抵抗)の観点でパターン形成を議論していたが,
sticky-endsの長さによる形成パターンの変化も十分に考えられる.
これは,先に説明したDNA独特の利点を利用したパターン制御であり,
DNAゲルカプセルのパターンがsticky-endsの長さにより制御可能になれば,
その研究は工学的により意義のあるものとなるだろうと考える.
近い将来にそのような研究が行われることは十分に考えられ,
今回作成した数理モデルにsticky-endsの長さに対応する拡張を施すこすことができれば,
そのIn Vitroの実験をシミュレーション実験がサポートしていくのではないかと期待する.
