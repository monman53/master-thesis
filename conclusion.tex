\chapter{結論}

自然界には自己組織化現象に由来する様々なパターン形成現象が存在し,
我々はそれらを数理的に解析することで基礎的な研究に限らず工学的応用も行ってきた.
近年,DNA分子の自己組織化現象を応用したDNAナノテクノロジーが盛んに研究され,
新しいナノ・マイクロスケールのマテリアルやパターン形成制御法として確立されつつある.
DNA分子の利点はそのプログラマビリティにあり,
他の物質では困難な分子構造や物性などを人間が比較的容易に制御できる可能性を持つ.
本研究ではDNAナノテクノロジーの1つであるDNAゲルについて注目し,
DNAゲルが作り出す微小なゲルカプセルにみられるパターン形成メカニズムに関する研究を行った.

DNAゲルカプセルには大きく分けて,孔のようなパターンが形成される場合と,
比較的孔が少なく一様なカプセルが形成される場合の2つが存在する.
先行研究の実験結果と今回補助的に行ったIn Vitroの実験の結果などをふまえ,
カプセルにおける孔の形成が粘弾性相分離と呼ばれる現象に起因すると仮定した(第\ref{sec:dnagel}章).
私は,既存の粘弾性相分離の数理モデルを元に,DNAゲルカプセル用の新しい拡張モデルを構築し,
そのパターン形成の計算機実験を行った.
また,そのパターンについての定量的な議論を行うための解析手法についても開発を行い,
両者のコンピュータプログラムの実装を行った(第\ref{sec:model}章).

計算機実験の結果,
実際のDNAゲルカプセルで見られるパターン形成と非常に類似した現象を再現することに成功した.
また,得られた構造について形成された孔の大きさの分布を見ることで定量的な解析を行った,
温度や粘性抵抗のパラメータを変え,それらの条件依存性について議論した結果,
先行研究で得られていたパターン形成の傾向についての結果を肯定的に説明することができた(第\ref{sec:result}章).

上記の結果は,DNAゲルカプセルに見られるパターンの形成のメカニズムが
粘弾性相分離現象によるものである可能性を結論付け,
さらにそのパターン制御の可能性についても示唆を与えた.

Y-DNA分子はsticky-endsの長さを変更すると,その熱力学的特性の変化から
Y-DNA集合体全体の物性が変化するという結果が報告されている~\cite{sato2019sequence}.
本研究で扱ったDNAカプセルでは,
温度変化やカプセルの大きさ(粘性抵抗)の観点でパターン形成を議論していたが,
sticky-endsの長さによる形成パターンの変化も十分に考えられる.
これは,先に説明したDNA独特の利点を利用したパターン制御であり,
DNAゲルカプセルのパターンがsticky-endsの長さにより制御可能になれば,
その研究は工学的により意義のあるものとなるだろうと考える.
近い将来にそのような研究が行われることは十分に考えられ,
今回作成した数理モデルにsticky-endsの長さに対応する拡張を施すこすことができれば,
そのIn Vitroの実験を計算機実験がサポートしていくのではないかと期待する.

計算機実験の観点でいうと,本研究ではメゾスコピック領域よりも少し大きなスケールを扱った.
ナノスケール領域では分子動力学を用いた計算機実験が盛んに行われており,
マイクロスケール領域ではレオロジーや流体力学などを用いてやはり計算実験が盛んである.
しかし,メゾスコピック領域の計算機実験は十分確立されているとは言えず,
ナノスケールで実験するにはコストがかかりすぎ,
マイクロスケールでは表現が荒すぎるという状況であると考える.
このような観点で言うと,
本研究で扱ったモデルのスケールの領域の計算機実験の研究はさらに行われていくべきと考える.
特にDNAナノテクノロジーの分野においてはY-DNA分子のsticky-endsでわかるように,
連続的な物理定数ではなく塩基配列のような離散的な情報が物性に関与する場合がわかっており,
それらを考慮した新しい形式の数値計算法が生み出される可能性がある.

さらに,上で述べた離散的な情報が物性に関与するという事実は情報工学的にも興味深い.
離散的な塩基配列情報をナチュラルコンピューティングに応用したDNAコンピューティングという分野は存在するが,
それをマテリアルに応用する研究は多くない.
本研究で扱ったDNAゲルはまさにその応用の1つであるが,
その特徴がより強く現われるDNA液滴がある~\cite{sato2019sequence}.
DNA液滴ではDNA集合体が液体のように流動性を持ち,
sticky-endsの制御により液滴同士の融合分離を制御することができる.
このようにDNAナノテクノロジーと情報工学との別の観点での新しい融合が期待される.
