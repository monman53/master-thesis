\chapter{実験方法}
% 先行研究に置いて,蛍光観察などからゲルカプセルのパターンはDNAと水分子の2成分から構成されていると考察された.
%TODO 要出典
% この考察からパターン形成は2成分からなる相分離現象により引き起こされると考えた.
% 相分離とは1相からなる混合物が2つ以上の相に分離する現象のことで,
% 例えば油と水の分離や,
% 合金においてみられる現象である.
%TODO 要出典

\section{モデルの選定}
ケルカプセルのパターン形成の要因と考えられるものは,
Y-DNA分子,水分子,そして液滴を形成する脂質膜である.
ゲルカプセル作製における脂質膜は2種類の脂質から構成されているが~\cite{morita2017formation},
脂質同士は物性の違いから相分離し,
脂質膜にパターンが形成されることが知られている~\cite{yanagisawa2014multiple}.

そこで,ゲルカプセルにおけるパターン形成が脂質膜の相分離現象に依るかを確認する実験を行った.

\subsection{方法}
バルク状態のY-DNAゲルを作製し,
カプセル作製時と同じような孔のような構造を形成するかを確認する.

\subsection{結果}
脂質膜上ではなく,バルクのDNAゲルにおいても孔のパターンが形成された.

\subsection{結論}
脂質膜が存在しなくてもDNAゲル上に孔の形成がみられたことから,
このパターン形成はY-DNA分子の溶液固有の性質であると考えた.
これは,脂質膜の相分離現象では説明できない可能性を示唆しており,
通常の相分離モデルとは異なるモデルを考える必要があった.

\section{粘弾性相分離}
液体における通常の相分離では少数成分が液滴を形成し安定状態に落ち着くが,
少数成分を構成する分子が粘弾性力などの相互作用の影響で拡散が遅い場合,
少数成分が液滴を形成する前にネットワークを形成し多数成分が液滴を形成するという通常とは逆の状態が過渡期に観測される.
このような現象は粘弾性相分離と呼ばれ~\cite{tanaka2000viscoelastic},
本研究で注目しているDNAゲルカプセル上のパターンもこのモデルで説明できるのではないかと考えた.

% 実際,1つのY-DNA分子を考えてもその分子量は...で,水分子の...よりも非常に大きい.
% 水と油の相分離の場合,一般的な油の分子量は...程度である.
%TODO 田中先生の粘弾性相分離の論文
%TODO 図

\section{数理モデル}
今回,粘弾性相分離現象の数値解析を行うにあたりArakiらのモデル~\cite{araki2005simple}を用いた.
このモデルは2次元空間での粘弾性相分離を仮定し,
少数成分を疎視化するラグランジュ型のモデルである.

はじめ,粒子(疎視化点)は六方格子状に一様に配置され,
各粒子は隣接する6つの粒子と自然長$0$のバネで接続される.
この状態から,以下のランジュバン方程式を数値的に解いていく.

\begin{equation}
\zeta
\frac{d}{dt}
\vec{R}_i
=
-\frac{d}{d\vec{R}_i}
U_{tot}\{\vec{R}_i\}
+\vec{\xi}_i
.
\end{equation}

$\vec{R}_i$は粒子$i$の位置ベクトル,$\zeta$は粘性抵抗を意味する.
$\vec{\xi}_i$は以下の揺動散逸定理を満たす熱ゆらぎの項である.

\begin{eqnarray}
\langle\vec{\xi}_i\rangle &=& 0, \\
\langle\vec{\xi}_i(t):\vec{\xi}_j(t')\rangle &=& 2T\zeta\delta_{ij}\delta(t-t')\bm{I}.
\end{eqnarray}

$U_{tot}$は粒子$i$の総ポテンシャルを意味し,
次のように2つのポテンシャルの和で表される.

\begin{eqnarray}

\end{eqnarray}



\section{球面上への拡張}
カプセル状での粘弾性相分離現象をシミュレーションするために,
本研究では上記のモデルの球面上への拡張を行った.


\section{温度項の拡張}
実際のゲルカプセルの作製では温度は一定ではなく,
線形に温度を下げていくアニーリングが施される.
この状況を再現するために,温度項を次のように変更した.
