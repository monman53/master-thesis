\chapter{DNAゲルカプセル}

\section{DNAゲル}

本研究で扱うDNAゲルは,Y-DNAと呼ばれる人工的に合成したDNA分子を用いる~\cite{sato2019sequence}.
Y-DNA分子は3本の1本鎖DNAがY字形状にハイブリダイゼーションした構造をとる(図\ref{fig:ydna}b).
それぞれの1本鎖DNA分子の塩基配列は,
Y字形状をとるように事前にコンピュータソフトウェア~\cite{zadeh2011nupack}を用いて設計される.
3本の腕の先端は完全に相補鎖を形成せず,1本鎖の状態で突出している.
この1本鎖の部分の塩基配列は回文配列になっており,
他の分子の腕の先とハイブリダイゼーションしうる状態になっている.
我々はこの部分を``sticky-ends''と呼んでいる.
%TODO 図を入れる
Y-DNA分子同士がsticky-endを介して相互作用しネットワークを形成することで,ゲルが形成される(図\ref{fig:ydna}c).

\begin{figure}
\centering
\includesvg{image/ydna.svg}
\caption{Y-DNAの模式図.
    (a) Y-DNA分子を構成する3種類の一本鎖DNAの塩基配列.
        sticky-endsの部分は回文配列で,それぞれ互いに相補の配列である.
    (b) アニーリング操作により3本の一本鎖DNAはY字形状にハイブリダイゼーションする.
    (c) 溶液中でY-DNA分子はsticky-endsを介してネットワークを形成し,ゲル状態になる.}

\label{fig:ydna}
\end{figure}

\section{DNAゲルカプセル}

DNAゲルをカプセル状に形成したものがDNAゲルカプセルである.
