\chapter{結果}
\label{sec:result}

第\ref{sec:model}章で説明した計算機実験の結果とその解析結果について述べる.
平面上のモデルと球面上のモデルについて,
それぞれ温度変化の有無による計算結果を述べ,
最後に先行研究のDNAゲルカプセルとの比較し考察を行う.


\section{平面モデルの結果}


\subsection{温度変化なし}
まず,今回施した拡張を一切適用しないオリジナルのモデルの計算機実験結果について述べる.
計算点は縦40,横40の計1600点からなり,長さ$1$のバネで接続された状態から計算を開始する.
各定数については
$\zeta=12000$,$\epsilon=0.4$,$\sigma=0.25$,$\Delta t=1$,$T=0.45, E_0=6.5$の条件で実験を行った.

前半は全体に小さな孔構造が形成されており,
この孔の部分には拡散の速い物質が集合していると考えられる(図\ref{fig:result_2d_without_anearing}b).
時間が経過するにつれ形成された孔が成長し(図\ref{fig:result_2d_without_anearing}c),
系全体でネットワーク構造が形成される(図\ref{fig:result_2d_without_anearing}d).
ネットワーク構造を形成していた粒子は次第にまとまりはじめ(図\ref{fig:result_2d_without_anearing}e-g),
最終的には前半で見られた孔のパターンとは成分が完全に逆転し,
粒子が集合しクラスターを構成する状態で安定となる(図\ref{fig:result_2d_without_anearing}h).
\begin{figure}
    \centering
    \includesvg[scale=0.9]{image/result_2d_without_anearing.svg}
    \caption{
        拡張を施す前の平面モデルの計算結果の時間変化.
        $\zeta=12000$,$\epsilon=0.4$,$\sigma=0.25$,$\Delta t=1$,
        $T=0.45$,$E_0=6.5$.
        (a) $t=0$.
        (b) $t=10000$.
        (c) $t=20000$.
        (d) $t=30000$.
        (e) $t=40000$.
        (f) $t=50000$.
        (g) $t=60000$.
        (h) $t=70000$.
    }
    \label{fig:result_2d_without_anearing}
\end{figure}
これは粘弾性相分離現象の一連の過程と類似しており,元のモデルが表現していた現象である.


\subsection{温度変化あり}
次に,温度変化の拡張を行った結果について述べる.
温度$T$以外の条件を先の温度変化なしの実験と同じにし,
温度$T$を式(\ref{eq:thermal})のように変更し対照実験を行う.
ただし$T_{max}=0.45$,$t_{max}=80000$とし,
この値は先の実験における$T$や計算時間と同じであるので,
両者の異なる点は$T$の温度が線形に小さくなっていく点のみである.

前半は温度変化のない場合と同じように全体に孔の構造が形成される(図\ref{fig:result_2d_with_anearing}a-d).
\begin{figure}
    \centering
    \includesvg[scale=0.9]{image/result_2d_with_anearing.svg}
    \caption{
        温度変化の拡張(式(\ref{eq:thermal}))を施した平面モデルの計算結果の時間変化.
        $\zeta=12000$,$\epsilon=0.4$,$\sigma=0.25$,$\Delta t=1$,
        $T_{max}=0.45$,$E_0=6.5$, $t_{max}=80000$.
        (a) $t=0$.
        (b) $t=10000$.
        (c) $t=20000$.
        (d) $t=30000$.
        (e) $t=40000$.
        (f) $t=50000$.
        (g) $t=60000$.
        (h) $t=70000$.
    }
    \label{fig:result_2d_with_anearing}
\end{figure}
後半はその孔が成長するものの,ネットワーク構造を保ったまま安定状態になる(図\ref{fig:result_2d_with_anearing}e-h).
これは温度が下がることにより揺動力やバネの切断確率が小さくなることに起因すると考えられる.

実際のDNAカプセルゲルも,
通常の粘弾性相分離のように構成成分がクラスターを形成するまで相分離が進行せず,
この実験で得られた結果のように途中ネットワーク構造を形成する段階でゲル化し安定化していると考えられる.


\subsubsection{アニーリング速度依存性}
アニーリング速度を変えた場合のパターンの違いについて述べる.
アニーリング速度を変えて実験するためには,
式(\ref{eq:thermal})より$t_{max}$を変更すれば良い.
$t_{max}$が小さいほどアニーリング速度が速いことになる.

アニーリング速度以外の条件は先の実験と同じに設定し,
$t_{max}=20000$,$t_{max}=40000$,$t_{max}=60000$,$t_{max}=80000$の場合について
それぞれ実験を行った(図\ref{fig:result_2d_anearing_speed}).
実験の結果,アニーリング速度により構造に大きな違いが現れることが確認された.
アニーリング速度が速いほど孔は形成されにくい傾向にあり,
形成されたとしても非常に小さいことが確認できる(図\ref{fig:result_2d_anearing_speed}a).
\begin{figure}
    \centering
    \includesvg{image/result_2d_anearing_speed.svg}
    \caption{
        平面モデルにおける構造のアニーリング速度依存性.
        温度変化ありのモデルにおいてアニーリング速度を変えて実験するためには$t_{max}$を変更する.
        式(\ref{eq:thermal})より,$t_{max}$が小さいほどアニーリング速度が速い.
        画像は$t=t_{max}$のときの構造のCGである.
        (a) $t_{max}=20000$.
        (b) $t_{max}=40000$.
        (c) $t_{max}=60000$.
        (d) $t_{max}=80000$.
    }
    \label{fig:result_2d_anearing_speed}
\end{figure}
これはバネの切れやすい温度の高い状態にある時間が少ないためであると考えられる.

孔の大きさの分布などの定量的な解析については,
球面モデルの実験結果において議論する.


\section{球面モデルの結果}
次に,モデルを球面上に拡張した場合の結果を述べる.
正二十面体のそれぞれの面を100等分して構築した,1002点からなるGeodesic Gridを初期状態とする.
各定数については平面モデルでの実験と同じである.
また,各条件につき8サンプルを計算し解析に用いた.


\subsection{温度変化なし}
まず,温度変化を行わない条件で実験を行った.
平面モデルの結果と同様に,
孔の成長とクラスターの形成が球面上で確認され
(図\ref{fig:result_sphere_without_anearing}),
\begin{figure}
    \centering
    \includesvg[scale=0.9]{image/result_sphere_without_anearing.svg}
    \caption{
        温度変化の無い球面モデルの計算結果の時間変化.
        $\zeta=12000$,$\epsilon=0.4$,$\sigma=0.25$,$\Delta t=1$,
        $T=0.45$,$E_0=6.5$.
        (a) $t=0$.
        (b) $t=10000$.
        (c) $t=20000$.
        (d) $t=30000$.
        (e) $t=40000$.
        (f) $t=50000$.
        (g) $t=60000$.
        (h) $t=70000$.
    }
    \label{fig:result_sphere_without_anearing}
\end{figure}
球面上へのモデルの拡張は正しく行われたと考えられる.


\subsection{温度変化あり}
次に,温度変化ありの場合の結果を図\ref{fig:result_sphere_with_anearing}に示す.
温度変化の条件は平面モデルの温度変化ありの実験と同じ
$T_{max}=0.45$,$t_{max}=80000$で行った.

\begin{figure}
    \centering
    \includesvg[scale=0.9]{image/result_sphere_with_anearing.svg}
    \caption{
        温度変化の拡張(式(\ref{eq:thermal}))を施した球面モデルの計算結果の時間変化.
        $\zeta=12000$,$\epsilon=0.4$,$\sigma=0.25$,$\Delta t=1$,
        $T_{max}=0.45$,$E_0=6.5$,$t_{max}=80000$.
        (a) $t=0$.
        (b) $t=10000$.
        (c) $t=20000$.
        (d) $t=30000$.
        (e) $t=40000$.
        (f) $t=50000$.
        (g) $t=60000$.
        (h) $t=70000$.
    }
    \label{fig:result_sphere_with_anearing}
\end{figure}
平面モデルと同じように,
初めに形成された小さな孔が次第に成長する様子が確認された.

実際のDNAゲルカプセルの実験ではアニーリング操作が行われており,
温度変化ありのこの計算機実験が本研究の中心である.
アニーリング速度と粘性抵抗の2種類のパラメータを変更した場合の
最終状態のパターンに関して詳細な解析を行い,
先行研究の結果との比較へ繋げる.


\subsubsection{アニーリング速度依存性}
粘性抵抗$\zeta$の値を$12000$に固定し,
$t_{max}$の値を変えて実験を行った.
$t=t_{max}$の構造の解析を行ったところ,
アニーリング速度が速いほど($t_{max}$が小さいほど)孔の形成がされにくいことがわかった
(図\ref{fig:result_sphere_anearing_speed_comb}).
\begin{figure}
    \centering
    \includesvg{image/result_sphere_anearing_speed_comb.svg}
    \caption{
        球面モデルにおける構造のアニーリング速度依存性.
        温度変化ありのモデルにおいてアニーリング速度を変えて実験するためには$t_{max}$を変更する.
        式(\ref{eq:thermal})より,$t_{max}$が小さいほどアニーリング速度が速い.
        上の画像は,$\zeta=12000$,$t=t_{max}$のときの構造のCGである.
        下のヒストグラムは上の構造の解析結果で,
        横軸は正規化された孔の径,縦軸は頻度(8サンプルの合計個数)である.
        右上の数字は関数$y=ae^{-bx}$で回帰したときの$b$の値で,
        この値が大きいほど孔が形成されにくい.
        アニーリング速度が遅いほど($t_{max}$が大きいほど)
        分布は右側にシフトしており孔が成長しやすいと言える.
        $\zeta=12000$.
        (a) $t_{max}=80000$.
        (b) $t_{max}=60000$.
        (c) $t_{max}=40000$.
        (d) $t_{max}=20000$.
    }
    \label{fig:result_sphere_anearing_speed_comb}
\end{figure}


\subsubsection{粘性抵抗依存性}
次に$t_{max}$の値を$60000$に固定し,
粘性抵抗$\zeta$の値を変えて実験を行った.
$t=t_{max}$の構造の解析を行ったところ,
粘性抵抗$\zeta$が大きいほど孔の形成がされにくいことがわかった
(図\ref{fig:result_sphere_friction_constant_comb}).
\begin{figure}
    \centering
    \includesvg{image/result_sphere_friction_constant_comb.svg}
    \caption{
        球面モデルにおける構造の粘性抵抗$\zeta$依存性.
        上の画像は,$t=t_{max}=60000$のときの構造のCGである.
        下のヒストグラムは上の構造の解析結果で,
        横軸は正規化された孔の径,縦軸は頻度(8サンプルの合計個数)である.
        右上の数字は関数$y=ae^{-bx}$で回帰したときの$b$の値で,
        この値が大きいほど孔が形成されにくい.
        $\zeta$が小さいほど分布は右側にシフトしており孔が成長しやすいと言える.
        $t_{max}=60000$
        (a) $\zeta=20000$.
        (b) $\zeta=16000$.
        (c) $\zeta=12000$.
        (d) $\zeta=8000$.
    }
    \label{fig:result_sphere_friction_constant_comb}
\end{figure}


\subsection{両者のまとめ}
以上でアニーリング速度と粘性抵抗の両者の依存性について述べた.
ここで,本研究で用いた全条件の結果を図\ref{fig:result_sphere_all}に示す.
\begin{figure}
    \centering
    \includesvg[scale=0.9]{image/result_sphere_all.svg}
    \caption{
        本研究で用いた全てのアニーリング速度$t_{max}$と
        粘性抵抗$\zeta$の条件対に対する計算機実験の結果のCG.
    }
\label{fig:result_sphere_all}
\end{figure}
また,その孔の分布を図\ref{fig:result_sphere_all_hist}に示す.
\begin{figure}
    \centering
    \includesvg[scale=0.9]{image/result_sphere_all_hist.svg}
    \caption{
        本研究で用いた全てのアニーリング速度$t_{max}$と
        粘性抵抗$\zeta$の条件対に対する計算機実験の結果の解析結果.
        各ヒストグラムの横軸は正規化された孔の径,縦軸は頻度(8サンプルの合計個数)である.
        右上の数字は関数$y=ae^{-bx}$で回帰したときの$b$の値で,
        この値が大きいほど孔が形成されにくい.
    }
    \label{fig:result_sphere_all_hist}
\end{figure}


\section{考察}
本実験では球面上の粘弾性相分離現象による孔構造のできやすさについて,
アニーリング速度と粘性抵抗の2つの観点から結果を得た.
この節では,
これらの結果について実際のDNAゲルカプセルのパターン形成の実験データと対応させながら考察する.

先行研究ではDNAゲルカプセルのパターンを,
孔のあるヘテロ型と一様なホモ型の2つに分けて議論している.
アニーリング速度とカプセルの大きさに対応する
両型の形成割合に関する統計量が実験により得られており(図\ref{fig:result_moritasan}a),
\begin{figure}
    \centering
    \includesvg{image/result_moritasan.svg}
    \caption{
        計算機実験の結果と先行研究の結果の比較.
        孔の形成されにくさの傾向が一致していることがわかる.
        (a) 先行研究の結果.
            左はアニーリング速度依存性,右は球の半径依存性.
            黄色が孔のあるヘテロ型,青色が孔の無いホモ型(一番右の凡例を参照されたい).
        (b) 計算機実験の結果.
            左はアニーリング速度依存性($\zeta=12000$),右は粘性抵抗依存性($t_{max}=60000$).
            どちらも縦軸は$b$の値であり,この値が大きいほど孔が形成されにくい.
            (先行研究の結果と数理モデルのパラメータを対応させるために
            横軸が逆向きなっていることに注意されたい.)
    }
    \label{fig:result_moritasan}
\end{figure}
それらのデータと本計算機実験の解析データとを比較した.


\subsection{アニーリング速度依存性}
先行研究の結果によれば,
アニーリング速度が速いほどホモ型のDNAゲルカプセルが形成されやすく(図\ref{fig:result_moritasan}a左),
この結果は計算機実験の結果(図\ref{fig:result_sphere_anearing_speed_comb})と一致する.
系の温度がsticky-endsの融解温度を下回るとY-DNA分子は互いに接続し合い,
完全にゲル化してしまう\cite{sato2019sequence}.
相分離現象が起こるためには物質の流動が必要であるが,
一度ゲル化してしまうと相分離がそれ以上進まないと考えられる.
したがってアニーリング速度が遅いほど,
つまりゲル化するまでの比較的高温の時間が長いほど孔のパターンは形成されやすいと考えられる.

\subsection{カプセルの半径依存性と粘性依存性の関係}
先行研究のもう一方の結果であるカプセルの大きさの依存性に関しては,
直接カプセルの大きさを変更する数値実験を行っていないため,
本モデルを使い次の仮説を検証する形で考察を行う.

第\ref{sec:dnagel}章で述べたとおりDNAゲルカプセルは球状の液滴の内面に形成されるが,
仮に全てのDNA分子が内面に集積しているとすると,
その総量は球の体積に比例するが,
集積するのは球面(面積)であるからその厚さは球の半径に比例することがわかる.
したがって,液滴の大きさが異なればDNAのゲル層の厚さも異なると考えられる.
半径$R$の理想的なDNA溶液の液滴を用いてこれを確認する.
溶液の濃度を$\rho$とし,全てのDNA分子が界面に濃度$\rho'$で均一に集積することを考える.
集積したDNAの層の厚さを$d$とすると,集積の前後で分子の総数は保存されるため
\begin{equation}
    \frac{4}{3}\pi R^3 \rho = \left(\frac{4}{3}\pi R^3-\frac{4}{3}\pi(R-d)^3\right)\rho'
\end{equation}
が成りたつ.
これを$d$について解くと
\begin{equation}
    d = R\left(1-\sqrt[3]{\frac{\rho'-\rho}{\rho'}}\right)
\end{equation}
となり,確かに$d$は$R$に比例することがわかる.

脂質と接するDNA分子は電気的相互作用により脂質膜に吸着するが,
その上に集積してくDNA分子はsticky-endsを介して集合体を形成する.
よって厚く集積するほど脂質膜に吸着しているDNAの割合は全体に対して小さくなり,
界面上での流動性が高くなる.
以上の議論から,
直径が大きい液滴ほど内面のDNAの流動性は高くなるという仮説をたてた(図~\ref{fig:size_and_friction}).
\begin{figure}
    \centering
    \includesvg{image/size_and_friction.svg}
    \caption{
        DNAゲルカプセル作成時におけるDNA溶液の液滴の大きさと粘性抵抗の関係についての模式図.
        (a) 液滴の直径が大きい場合.
            脂質に近いDNAは電気的な相互作用により引き合うが,
            その上に集積したDNA分子にはほとんど働かず,
            より流動性があると考えられる.
        (b) 液滴の直径が小さい場合.
    }
    \label{fig:size_and_friction}
\end{figure}
ここで言う流動性はモデルにおける粘性抵抗(式(\ref{eq:main}))と対応すると考えることで,
計算機実験の結果における粘性抵抗依存性
(図\ref{fig:result_sphere_friction_constant_comb})
と比較することでこの仮説を肯定的に検証することができる.

先行研究によれば,
液滴の大きさが大きいほどヘテロ型のDNAゲルカプセルが形成されやすく(図\ref{fig:result_moritasan}a右), 
この結果は先の仮説を合わせると計算機実験の結果と一致する.
