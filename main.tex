% \documentclass[12pt, a4paper]{jsreport}
\documentclass[a4paper,12pt,oneside,openany,report]{jsbook}
% \usepackage[left=30mm,right=30mm,top=25mm,bottom=25mm]{geometry}
%================
% パッケージ・マクロ
%================
% テンプレート
\usepackage{cscover}

% for hyperref
\usepackage[a4paper,dvipdfmx,pdfdisplaydoctitle=true,%
    bookmarks=true,bookmarksnumbered=true,bookmarkstype=toc,bookmarksopen=true,%
    pdftitle={DNAゲルのミクロパターン形成の数理モデルの構築と数値的な解析},%
    pdfauthor={阪本哲郎}%
    ]{hyperref}
\usepackage{pxjahyper}
\hypersetup{
    % pdfborder = {0 0 0},
}
% フォント
\AtBeginDvi{\special{pdf:mapfile ptex-ipa.map}}
% 数学関連
\usepackage{amsmath,amsthm,amssymb,ascmac}
\usepackage{bm}
% 画像
\usepackage{svg}
% 単位
\usepackage{siunitx}
% 引用数字を行頭にしない
\usepackage[nobreak]{cite}
% ヘッダ
% \pagestyle{headings}



\renewcommand{\bibname}{参考文献}
\setcounter{tocdepth}{2}
\pagestyle{plain}

\newcommand{\TODO}[1]{\textbf{[TODO: #1]}}
%\renewcommand{\TODO}[1]{}

%================
% メタ情報
%================

\thesistype{修士論文}
%\thesistype{博士論文}
% \title{学位論文の\\体裁に関する研究}
\title{DNAゲルのミクロパターン形成の\\数理モデルの構築と数値的な解析}
\author{阪本 哲郎}
\studentid{18M31094}
\date{2020-01-27}
\affiliation{%
  東京工業大学\\
  情報理工学院\\
  知能情報コース
  % 情報工学コース
}

\supervisorname{指導教員}
\supervisor{瀧ノ上 正浩}
% \dsupervisorname{副査教員}
% \dsupervisor{山村 雅幸}
% \dsupervisorname{副査教員}
% \dsupervisor{関嶋 政和}



\begin{document}

%================
% 表紙
%================
\frontmatter
\maketitle

%================
% 概要
%================
\chapter*{概要}

概要

\clearpage

%================
% 目次
%================
\setcounter{tocdepth}{2}
\tableofcontents
% \listoffigures


%================
% 内容
%================
\mainmatter

%----------------
% イントロ
%----------------
\chapter{緒論}

\section{自然界のパターン形成}
私達の身の回りの自然界には非常に多種多様なパターンを見出すことができる.
雪の結晶,山や雲の模様,土や塗料のひび割れなどの見慣れたものから,
植物の葉や枝に見られるフラクタル構造や動物の体表の特徴的な模様など,
% TODO ref
生物が自ら作り出すものもある.
パターン形成は目に見えるマクロな世界に限らず,
ミクロな世界でも至るところで行われている.
特に生物の構成要素である細胞では,パターン形成が生命現象に深く関わっていることがわかり始めている.
パターン形成のメカニズムを探ることでその現象の原理や本質を理解できる場合があり,
その行為は基礎科学的にみて重要である.
例えば動物の体表などは反応拡散系というモデルにより良く理解されている.
% TODO ref

さらに自然界のパターン形成の理解は工学的にも有用である.
% 植物の葉脈の例でいうと,そのフラクタル次元を用いた種の同定方法が提案されている.
% TODO ref
人間はその現象を応用し利用してきた. 
% TODO 推敲
自然界にみられるパターン形成は自己組織的に行われることが多く,
この特徴はマテリアルの分野,
特にボトムアップの形成が困難な微小な構造形成に力を発揮する.
スポンジ,特殊表面加工など,例を挙げると枚挙に暇がない.
% TODO もっと例を


\section{DNAナノテクノロジー}
先に挙げた例はいずれも重要な技術だが,
化学合成などの従来手法を用いた分子スケールの物質の制御は未だに困難を極める.
したがって,開発には多くの試行錯誤が必要で時間やコストが無視できない.
そこで,人間がプログラマブルなデジタルコンピュータを手に入れたように,
マテリアルの分野においても制御の容易なプログラマブルな物質が求められている.

近年,そのような物質としてDNAが注目されている.
DNAを用いたナノスケールの工学的応用はDNAナノテクノロジーと呼ばれ,
盛んに研究が行われている.
DNAは生物の遺伝情報を担う物質として知られているが,
DNAナノテクノロジーではこれをマテリアルとしても利用する.
% TODO DNAの説明を詳しく
DNAのこの特徴は,先のプログラマブルな物質としての条件を満たす.

DNAナノテクノロジーの応用の1つにDNAゲルがある.
我々はこのゲルを用いたマイクロスケールの構造体の作製に成功している.
主に人工細胞の要素技術への応用が考えられており,
その1つは細胞をみたてた脂質二重膜でできたマイクロスケールの小胞の内側にDNAゲルを裏打ちさせる技術である.
DNAゲルの裏打ち構造により,
この小胞は過酷な浸透圧変化に耐えられることが知られている~\cite{kurokawa2017dna}.
また,DNAゲル自身によるカプセル構造の作製も行われている~\cite{morita2017formation}.
遺伝情報をコーディングしたDNAゲルを使用し,
そのゲルの収縮を利用してRAN転写効率を制御する方法が提案されている~\cite{watanabe}.
これらのDNAゲルによるマイクロ構造体はその生体親和性からドラッグデリバリーへの応用も期待される.

このマイクロスケールの構造体の作製においても,自己組織的なパターン形成が応用されている.
1つはゲルを構成するDNA分子形成であるナノスケールのパターン形成で,
もう1つはゲルのマクロな形状を決定するパターン形成である.
前者は熱力学の応用により予測や制御が可能になりつつあるが~\cite{zadeh2011nupack}, 
後者についてはメカニズムが明確に理解されていない.


\section{研究目的}
本研究では特にDNAゲルのパターン形成に注目し,
そのメカニズムを主に数値解析により明らかにすることである.
そこで,
DNAゲルカプセルに見られるパターン形成は粘弾性相分離現象というモデルで説明できると仮定し,
カプセルを見立てた球面上でその数値計算を行った.
その結果と,先行研究で得られていた In vitro の実験結果とを定量的に比較した.
(計算機を用いた実験の利点を記述) %TODO
%In vitro の実験を行う上で予測をたてることに

\section{論文構成}
%TODO 後で変更になる
本論文は全5章から構成される.
第1章では,
緒論として自然界のパターン形成のその工学的有用性,
またDNAナノテクノロジーと本研究との関連について述べた.
第2章では,
本研究で扱うDNAゲルカプセルについての詳細を述べる.
この内容はMoritaらの研究に基づく~\cite{morita2017formation}.
また,パターン形成が脂質に寄らないことを裏付ける補助実験について説明する.
第3章では,
実際に私が行った実験の手法について述べる.
はじめに,DNAゲルカプセルのパターン形成のシミュレーションに使用する数理モデルの選定を行う.
次に数理モデルにアニーリング操作を反映する拡張を行い,
最後にカプセルの形成を再現するために全体を球面上に拡張する.
第4章では,
数値実験の結果を述べる.
最後に第5章で結論と考察を述べる.


%----------------
% より詳しいイントロ
%----------------
\chapter{DNAゲルカプセル}

\section{DNAゲル}

本研究で扱うDNAゲルは,Y-DNAと呼ばれる人工的に合成したDNA分子を用いる~\cite{sato2019sequence}.
Y-DNA分子は3本の1本鎖DNAがY字形状にハイブリダイゼーションした構造をとる.
それぞれの1本鎖DNA分子の塩基配列は,
Y字形状をとるように事前にコンピュータソフトウェア~\cite{zadeh2011nupack}を用いて設計される.
3本の腕の先端は完全に相補鎖を形成せず,1本鎖の状態で突出している.
この1本鎖の部分の塩基配列は回文配列になっており,
他の分子の腕の先とハイブリダイゼーションしうる状態になっている.
我々はこの部分を``sticky-ends''と呼んでいる.
%TODO 図を入れる
Y-DNA分子はsticky-endを介して相互作用し,
分子同士のネットワークを形成することでゲルを形成すると考えられている.

\section{DNAゲルカプセル}

DNAゲルをカプセル状に形成したものがDNAゲルカプセルである.



%----------------
% モデルの説明
%----------------
\chapter{数理モデル}
\label{sec:model}

本研究では主にDNAゲルカプセルのパターン形成メカニズムを解明するために数値シミュレーションの実験を行った.
そのために粘弾性相分離という特殊な相分離現象のモデルを元に,
DNAゲルカプセルに対応するような拡張モデルを開発し,
そのシミュレーションプログラムを実装した.
また,得られた構造を実際のDNAゲルカプセルと比較するための解析手法についても新たに考案した.
本章では拡張モデルの説明と解析手法の詳細について述べる.

\section{数理モデル}

\subsection{粘弾性相分離の数理モデル}
今回,粘弾性相分離現象の数値解析を行うにあたりArakiらのモデル~\cite{araki2005simple}を用いた.
このモデルでは2次元空間での粘弾性相分離を仮定し,
少数成分を疎視化するラグランジュ型のモデルである.

はじめに粒子(疎視化点)は六方格子状に一様に配置され,
各粒子は隣接する6つの粒子と自然長$0$のバネで接続される.
この状態から,以下のランジュバン方程式を数値的に解いていく.


\begin{figure}
\centering
\includesvg{image/model_2d.svg}
\caption{
    Araki~\cite{araki2005simple}らのモデルの概要.
    (a) 初期状態.計算点の粒子が六方格子上に等間隔に配置され,
        隣接する6つの粒子と自然長$0$で長さ$l_0$のバネにより接続される.
        空間は周期境界条件が適用される.
    (b) 初期配置の拡大図.
    (c) シミュレーションの時間発展の様子.
        粒子の熱ゆらぎと共に,
        接続されていたバネが次第に切断されていくことで粘弾性相分離現象が再現される.
}
\label{fig:model_2d}
\end{figure}


\begin{equation}
\label{eq:main}
\zeta
\frac{d}{dt}
\vec{R}_i
=
-\frac{\partial}{\partial\vec{R}_i}
U_{tot}\{\vec{R}_i\}
+\vec{\xi}_i
.
\end{equation}

$\vec{R}_i$は粒子$i$の位置ベクトル,$\zeta$は粘性抵抗を意味し粘弾性の粘性の部分を表現する.
$\vec{\xi}_i$は以下の揺動散逸定理を満たす粒子iにおける熱ゆらぎの項である.

\begin{eqnarray}
\label{eq:langevin0}
\langle\vec{\xi}_i\rangle &=& 0, \\
\label{eq:langevin1}
\langle\vec{\xi}_i(t):\vec{\xi}_j(t')\rangle &=& 2T\zeta\delta_{ij}\delta(t-t')\bm{I}.
\end{eqnarray}

$U_{tot}$は粒子$i$の総ポテンシャルであり,
次の2つのポテンシャルの和で表される.

\begin{eqnarray}
U_{tot}\{\vec{R}_i\}
&=
 U_{LJ}\{\vec{R}_i\}
+U_{sp}\{\vec{R}_i\}.
\end{eqnarray}

$U_{LJ}$,$U_{sp}$はそれぞれレナード・ジョーンズポテンシャルとバネによる調和ポテンシャルであり,
定義は次の通りである.

\begin{eqnarray}
U_{LJ}\{\vec{R}_i\}
&=&
4\epsilon\sum_{j\neq i}
\left\{
\left(
\frac{\sigma}{\bigl|\vec{R}_j-\vec{R}_i\bigr|}
\right)^{12}
-
\left(
\frac{\sigma}{\bigl|\vec{R}_j-\vec{R}_i\bigr|}
\right)^{6}
\right\},
\\
U_{sp}\{\vec{R}_i\}
&=&
\frac{1}{2}
\kappa
\sideset{}{'}\sum_{j\neq i}
\bigl|\vec{R}_j-\vec{R}_i\bigr|^{2}.
\end{eqnarray}

$\epsilon$,$\sigma$,$\kappa$はそれぞれポテンシャルを特徴づける定数であり,
調和ポテンシャルにおける記号$\sideset{}{'}\sum$はバネにより接続された粒子間での和を意味する.

初期状態で接続されていたバネは,時間が経つにつれランダムに切断されていく.
その切断確率$p$は温度$T$とバネの長$l$に依存する形で定義され,
各計算ステップで全バネに対して切断するかしないかの処理がなされる.
このバネが粘弾性の弾性力の部分を表現する.

\begin{eqnarray}
p(l)
&=&
t_0^{-1}
\exp\left(-\frac{\Delta E(l)}{T}\right)
,\\
\Delta E(l)
&=&
E_0-\frac{\kappa l^2}{2}
\end{eqnarray}

具体的には,$0$から$1$の一様乱数$r$を発生させ,$r<p(l)$ならバネを切断する.
$t_0$は数値計算の1ステップの時間で,$E_0$はバネの切れやすさを意味する定数である.
一度切断されたバネは二度と繋がることはなく,粒子間で新たにバネが接続されることもない.
このバネの切断が相分離を表現する.


\subsection{計算結果のポリゴン化と配色の変更}

元のモデルでは計算結果は頂点と辺(バネ)のみであるが,
相分離により形成された孔の解析(次節参照)を行うためにポリゴン化を行った.
ポリゴンを構成する三角形は,ある計算点と隣接する計算点で作られる三角形を全て用いた.
また,計算結果をわかりやすくするために配色を変更した(図\ref{fig:model_2d}).

\begin{figure}
\centering
\includesvg{image/polygon.svg}
\caption{
    計算結果のポリゴン化と配色の変更.
    (a) 変更前の計算結果のCG.
    (b) 同じ計算結果に対する変更の適用後のCG.
}
\label{fig:model_2d}
\end{figure}


\subsection{温度項の拡張}
実際のゲルカプセルの作製では温度は一定ではなく,
線形に温度を下げていくアニーリングが施される.
この状況を再現するために,温度項を次のように変更した.

\begin{eqnarray}
\label{eq:thermal}
T = T_{max}(1-\frac{t}{t_{max}})
\end{eqnarray}

$T_{max}$は初期状態の温度で,$t_{max}$はシミュレーションの終了時間である.
つまり$t=0$で$T=T_{max}$,
$t=t_{max}$で$T=0$になるような線形アニーリングを再現している.


\subsection{球面上への拡張}
カプセル状での粘弾性相分離現象をシミュレーションするために,
本研究では上記のモデルの球面上への拡張を行った\ref{fig:model_sphere}.

まず,ある大きさのGeodesic Grid~\cite{Geodesic}を用意し,
その頂点に粒子を配置し辺の部分をバネとすることで初期状態とする.
この状態から式\ref{eq:main}運動方程式を解き,
移動した$R_i$を球面上に射影し直すことで常に粒子が球面上にとどまるように計算を繰り返していく.
他の処理に関しては元のモデルと同じである.

\begin{figure}
\centering
\includesvg{image/model_sphere.svg}
\caption{
    球面モデルの模式図.
    (a) Geodesic Grid~\cite{Geodesic}の構築方法.
        正二十面体の面を三角形に等分割し,頂点を球面に射影することで構築する.
        球面モデルにおいては,この構築されたグリッドを初期状態としてシミュレーションを行う.
    (b) 球面モデルにおけるシミュレーションの時間発展.
}
\label{fig:model_sphere}
\end{figure}



%TODO webpage の ref



\section{実装}
本モデルがラグランジュ型のモデルであり,
バネの切断などの特殊な処理が必要であることから,
汎用のシミュレーションソフトウェアでの実験は困難であると考えた.
しかし,モデルは単純であり実装は困難ではないと判断したためシミュレーションプログラムは自ら実装した.

まず,オイラー法による計算を考え,運動方程式(\ref{eq:main})を次のように離散化した.
\begin{eqnarray}
    \vec{R}_i(t+\Delta t) &=& 
    \vec{R}(t)_i
    -\frac{1}{\zeta}\frac{\partial}{\partial\vec{R}_i(t)}U(\vec{R}_i(t))\Delta t
    +\chi\sqrt{\frac{2k_B T}{\zeta}\Delta t}
    .
\end{eqnarray}

次に,各物理量をそれぞれ
$x=l_0 \tilde{x}$,
$t=t_0 \tilde{t}$,
$\zeta=\kappa t_0 \tilde{\zeta}$,
$E=\frac{1}{2}\kappa\l_0^2\tilde{E}$
でスケールし無次元化した.
以下,特に記述がなければ物理量は無次元化されたものとする.


プログラムはC++言語で記述し,数値計算には線形代数ライブラリEigen~\cite{eigenweb}を使用した.
データの可視化にはParaView~\cite{paraview}とPOV-Ray~\cite{povray}をそれぞれ用いた.
計算はラップトップPC上(Intel Core i5-7300U CPU)で並列実行した.


\section{解析}
シミュレーションで得られた構造を定量的に解析する必要がある.
本研究では構造の特徴として孔の大きさの分布を考えた.

\subsection{孔の検出アルゴリズム}

得られた構造のポリゴンとその構造に対して十分小さなグリッドを持つグリッドを重ね合わせ,
三角形が重なり合わないグリッドの連続領域を深さ優先探索などのアルゴリズムにより検出し,
その領域を孔とみなして面積などの特徴量を解析に用いる手法を開発した.


\begin{figure}
\centering
\includesvg{image/xor.svg}
\caption{
    孔の検出方法.
    検出したい構造とグリッドを重ね合わせ,
    両者の全ての三角形対に対して三角形の交差判定を行うことで,
    両者が重なり合わない部分,つまり孔を近似的に抽出することができる.
    (a) シミュレーションにより得られた構造のポリゴン.
    (b) 検出に用いる十分に細かいグリッド.
    (c) 両者を重ね合わせた状態.
    (d) 構造のポリゴンと交差しないグリッド上の三角形.
}
\label{fig:xor}
\end{figure}


孔の検出で行いたい処理は,
得られた構造のポリゴンの三角形とグリッドを構成するポリゴンの三角形との交差判定を繰り返し行うことで実現できる.
三角形の交差判定に必要な計算機幾何の基本的なアルゴリズムについて説明する.

\subsubsection{点と直線の位置関係}
2次元平面の2点が,ある直線を堺にどちら側に存在するかを議論する方法について説明する.
平面上の2つのベクトルの外積の値は,時計回りに演算された場合に正に,半時計回りに演算された場合に負になる.
この事実を用いると,直線上の2点$q_1$,$q_2$を選び,
調べたい点$p$からの2つのベクトル$a=q_1-p$,$b=q_2-p$の外積$(a)\times(b)$の符号を調べれば良い.
例えば2点$p_1$,$p_2$が直線を堺に同じ側に存在する場合は,この外積の符号が一致する.
逆に直線を堺に分けられている場合は外積の符号が一致しない.

\subsubsection{線分と線分の交差判定}
点と直線の位置関係を応用して,線分と線分の交差判定を行うことができる.
つまり線分と線分の交差は4回の外積演算により判定することができる.

\begin{figure}
\centering
\includesvg{image/segment_segment.svg}
\caption{
    点と直線の位置関係をベクトルの外積を用いて議論する方法と,
    それを応用した線分と線分の交差判定.
    (a) 直線上の2点$q_1$と$q_2$をこの順番を保ったまま$p$からの2本のベクトルを引き,
        その外積の符号を考える.
        $p$を直線に対して左右に移動するとその外積の符号は変化する.
    (b) 線分と線分の交差判定は,
        線分の2つの端点がもう一方の直線に対して反対側に存在することを2線分両方について言えば良い.
        つまり4回の外積演算で線分と線分の交差判定を行うことができる.
}
\label{fig:segment_segment}
\end{figure}

\subsubsection{点と三角形の交差判定}
三角形の3辺に対して,ある点が全て同じ側に存在する(3つの外積の符号が一致する)場合,
その点は三角形の内部にあることが言える.
つまり点が三角形に内包されるか否かを3回の外積演算により判定することができる.

\begin{figure}
\centering
\includesvg{image/point_triangle.svg}
\caption{
    点の三角形内外判定アルゴリズム.
    点$p$から点$q_1$,$q_2$(あるいは$q_2$,$q_3$や$q_3$,$q_1$)へのベクトルの外積を調べ,
    それら3つの値の符号を見ることにより判定できる.
    (a) 点が三角形の外側にある場合は外積の符号が一致しない.
        左2つは外積の値が負になるが,一番右のケースで正になる.
    (b) 点が内側にある場合は外積の符号が全て一致する.
}
\label{fig:point_triangle}
\end{figure}

\subsubsection{三角形と三角形の交差判定}
三角形と三角形の交差判定は,先に説明した線分と線分の交差判定と点と三角形の交差判定を複数回行うことで実現できる.
例えば,三角形と三角形の位置関係は図\ref{fig:triangle_triangle}にあるような場合が考えられる.
図\ref{fig:triangle_triangle}(c)のような場合に関しては線分と線分の交差判定が必要になる.
%TODO 図

\begin{figure}
\centering
\includesvg{image/triangle_triangle.svg}
\caption{
    三角形と三角形の位置関係の例.
    (a) 交差しない場合.
    (b) 一方の頂点が他方に内包される場合.
    (c) 一方がもう一方に完全に内包される場合.
    (d) 一方の頂点が他方に内包されないが三角形が交差している場合.
        (a)から(c)は三角形と点の交差判定を行うだけで判定できるが,
        (d)を検出するためは線分の交差判定を行う必要がある.
}
\label{fig:triangle_triangle}
\end{figure}

\subsubsection{球面上の三角形の交差判定}
2次元平面では正方格子を用いて孔の面積を算出することも可能であるが,
球面上で正方格子を考えることは非常に煩雑であり,
また今回はGeodesic Gridを利用しているため,
先に説明した三角形と三角形の交差判定を用いた.

球面上の三角形の交差判定についても,2次元平面で外積を考えたように,
級の中心を原点とする3つの位置ベクトルからなる行列の行列式の符号をみることで判定することができる.


\subsection{孔の検出例}
以上のアルゴリズムを用いた孔の検出例を図\ref{fig:hole_detection}に示す.
%TODO 図と共に

\begin{figure}
\centering
\includesvg{image/hole_detection.svg}
\caption{
    孔の検出例.
    わかりやすさのために画像では構造の半分のみを描画している.
    (a) 検出したい構造のポリゴン.
    (b) 検出に用いるGiodesic Grid.
    (c) 検出された孔.
}
\label{fig:hole_detection}
\end{figure}


%----------------
% 結果
%----------------
\chapter{結果}
\label{sec:result}

第\ref{sec:model}章で説明した計算機実験の結果とその解析結果について述べる.
平面上のモデルと球面上のモデルについて,
それぞれ温度変化の有無による計算結果を述べ,
最後に先行研究のDNAゲルカプセルとの比較し考察を行う.
%TODO 球面の影響についても言及?


\section{平面モデルの結果}


\subsection{温度変化なし}
まず,今回施した拡張を一切適用しないオリジナルのモデルを用いたシミュレーション結果について述べる.
計算点は縦40,横40の計1600点からなり,長さ$1$のバネで接続された状態から計算を開始する.
各定数については
$\zeta=12000$,$\epsilon=0.4$,$\sigma=0.25$,$\Delta t=1$,$T=0.45, E_0=6.5$の条件で実験を行った.

前半では全体に小さな孔構造が形成されており,
この孔の部分には拡散の速い物質が集合しているとみなすことができる(図\ref{fig:result_2d_without_anearing}b).
時間が経過するにつれ形成された孔が成長し(図\ref{fig:result_2d_without_anearing}c),
系全体でネットワーク構造が形成される(図\ref{fig:result_2d_without_anearing}d).
ネットワーク構造を形成していた粒子は次第にまとまりはじめ(図\ref{fig:result_2d_without_anearing}e-g),
最終的には前半で見られた孔のパターンとは成分が完全に逆転し,
粒子が集合しクラスターを構成する状態で安定となる(図\ref{fig:result_2d_without_anearing}h).
\begin{figure}
    \centering
    \includesvg[scale=0.9]{image/result_2d_without_anearing.svg}
    \caption{
        拡張を施す前の平面モデルの計算結果の時間変化.
        (a) $t=0$.
        (b) $t=10000$.
        (c) $t=20000$.
        (d) $t=30000$.
        (e) $t=40000$.
        (f) $t=50000$.
        (g) $t=60000$.
        (h) $t=70000$.
    }
    \label{fig:result_2d_without_anearing}
\end{figure}
これは粘弾性相分離現象の一連の過程と類似しており,元のモデルが表現していた現象である.


\subsection{温度変化あり}
次に,温度変化の拡張を行った結果について述べる.
温度$T$以外の条件を先の温度変化なしの実験と同じにし,
温度$T$を式\ref{eq:thermal}のように変更し対照実験を行う.
ただし$T_{max}=0.45$,$t_{max}=80000$とし,
この値は先の実験における$T$やシミュレーション時間と同じであるので,
両者の異なる点は$T$の温度が線形に小さくなっていく点のみである.

前半は温度変化のない場合と同じように全体に孔の構造が形成される(図\ref{fig:result_2d_with_anearing}a-d).
\begin{figure}
    \centering
    \includesvg[scale=0.9]{image/result_2d_with_anearing.svg}
    \caption{
        温度変化の拡張を施した平面モデルの計算結果の時間変化.
        (a) $t=0$.
        (b) $t=10000$.
        (c) $t=20000$.
        (d) $t=30000$.
        (e) $t=40000$.
        (f) $t=50000$.
        (g) $t=60000$.
        (h) $t=70000$.
    }
    \label{fig:result_2d_with_anearing}
\end{figure}
後半はその孔が成長するものの,ネットワーク構造を保ったまま安定状態になる(図\ref{fig:result_2d_with_anearing}e-h).
これは温度が下がることにより揺動力やバネの切断確率が小さくなることに起因する.

実際のDNAカプセルゲルも,
通常の粘弾性相分離のように構成成分がクラスターを形成するまで相分離が進行せず,
この実験で得られた結果のように途中ネットワーク構造を形成する段階でゲル化し安定化していると考えられる.


\subsubsection{アニーリング速度依存性}
アニーリング速度を変えた場合のパターンの違いについて述べる.
アニーリング速度を変えて実験するためには,
式\ref{eq:thermal}より$t_{max}$を変更すれば良い.
$t_{max}$が小さいほどアニーリング速度が速いことになる.

アニーリング速度以外の条件は先の実験と同じに設定し,
$t_{max}=20000$,$t_{max}=40000$,$t_{max}=60000$,$t_{max}=80000$の場合について実験を行った(図\ref{fig:result_2d_anearing_speed}).
実験の結果,アニーリング速度により構造に大きな違いが現れることが確認された.
アニーリング速度が大きいほど孔は形成されにくい傾向にあり,
形成されたとしても非常に小さいことが確認できる(図\ref{fig:result_2d_anearing_speed}(a)).
\begin{figure}
    \centering
    \includesvg{image/result_2d_anearing_speed.svg}
    \caption{
        平面モデルにおける構造のアニーリング速度依存性.
        温度変化ありのモデルにおいてアニーリング速度を変えて実験するためには$t_{max}$を変更する.
        式\ref{eq:thermal}より,$t_{max}$が小さいほどアニーリング速度が速い.
        画像は$t=t_{max}$のときの構造のCGである.
        (a) $t_{max}=20000$.
        (b) $t_{max}=40000$.
        (c) $t_{max}=60000$.
        (d) $t_{max}=80000$.
    }
    \label{fig:result_2d_anearing_speed}
\end{figure}
これはバネの切れやすい温度の高い状態にある時間が少ないためであると考えられる.

孔の大きさの分布などの定量的な解析については,
球面モデルの実験結果において議論する.


\section{球面モデルの結果}
次に,モデルを球面上に拡張した場合の結果を述べる.
正二十面体のそれぞれの面を100等分して構築した,1002点からなるGeodesic Gridを初期状態とする.
各定数については平面モデルでの実験と同じである.


\subsection{温度変化なし}
まず,温度変化を行わない条件で実験を行った.
平面モデルの結果と同様に,
孔の成長とクラスターの形成が球面上で確認され
(図\ref{fig:result_sphere_without_anearing}),
\begin{figure}
    \centering
    \includesvg[scale=0.9]{image/result_sphere_without_anearing.svg}
    \caption{
        温度変化の無い球面モデルの計算結果の時間変化.
        (a) $t=0$.
        (b) $t=10000$.
        (c) $t=20000$.
        (d) $t=30000$.
        (e) $t=40000$.
        (f) $t=50000$.
        (g) $t=60000$.
        (h) $t=70000$.
    }
    \label{fig:result_sphere_without_anearing}
\end{figure}
球面上へのモデルの拡張は正しく行われたと考えられる.


\subsection{温度変化あり}
次に,温度変化ありの場合の結果を図\ref{fig:result_sphere_with_anearing}に示す.
\begin{figure}
    \centering
    \includesvg[scale=0.9]{image/result_sphere_with_anearing.svg}
    \caption{
        温度変化の拡張を施した球面モデルの計算結果の時間変化.
        (a) $t=0$.
        (b) $t=10000$.
        (c) $t=20000$.
        (d) $t=30000$.
        (e) $t=40000$.
        (f) $t=50000$.
        (g) $t=60000$.
        (h) $t=70000$.
    }
    \label{fig:result_sphere_with_anearing}
\end{figure}
平面モデルと同じように,
初めに形成された小さな孔が次第に成長する様子が確認された.

実際のDNAゲルカプセルの実験ではアニーリング操作が行われており,
温度変化ありのこの計算機実験が本研究の中心である.
アニーリング速度と粘性抵抗の2種類のパラメータを変更した場合の
最終状態のパターンに関して詳細な解析を行い,
先行研究の結果との比較へ繋げる.


\subsubsection{アニーリング速度依存性}
粘性抵抗$\zeta$の値を$12000$に固定し,
$t_{max}$の値を変えて実験を行った.
$t=t_{max}$の構造の解析を行ったところ,
アニーリング速度が大きいほど($t_{max}$が小さいほど)孔の形成がされにくいことがわかった
(図\ref{fig:result_sphere_anearing_speed_comb}).
\begin{figure}
    \centering
    \includesvg{image/result_sphere_anearing_speed_comb.svg}
    \caption{
        球面モデルにおける構造のアニーリング速度依存性.
        温度変化ありのモデルにおいてアニーリング速度を変えて実験するためには$t_{max}$を変更する.
        式\ref{eq:thermal}より,$t_{max}$が小さいほどアニーリング速度が速い.
        上の画像は,$\zeta=12000$,$t=t_{max}$のときの構造のCGである.
        下のヒストグラムは上の構造の解析結果で,
        横軸は正規化された孔の径,縦軸は頻度(8サンプルの合計個数)である.
        右上の数字は関数$y=ae^{-bx}$で回帰したときの$b$の値で,
        この値が大きいほど孔が形成されにくい.
        アニーリング速度が遅いほど($t_{max}$が大きいほど)
        分布は右側にシフトしており孔が成長しやすいと言える.
        (a) $t_{max}=80000$.
        (b) $t_{max}=60000$.
        (c) $t_{max}=40000$.
        (d) $t_{max}=20000$.
    }
    \label{fig:result_sphere_anearing_speed_comb}
\end{figure}


\subsubsection{粘性抵抗依存性}
次に$t_{max}$の値を$60000$に固定し,
粘性抵抗$\zeta$の値を変えて実験を行った
$t=t_{max}$の構造の解析を行ったところ,
粘性抵抗$\zeta$が大きいほど孔の形成がされにくいことがわかった
(図\ref{fig:result_sphere_friction_constant_comb}).
\begin{figure}
    \centering
    \includesvg{image/result_sphere_friction_constant_comb.svg}
    \caption{
        球面モデルにおける構造の粘性抵抗$\zeta$依存性.
        上の画像は,$t=t_{max}=60000$のときの構造のCGである.
        下のヒストグラムは上の構造の解析結果で,
        横軸は正規化された孔の径,縦軸は頻度(8サンプルの合計個数)である.
        右上の数字は関数$y=ae^{-bx}$で回帰したときの$b$の値で,
        この値が大きいほど孔が形成されにくい.
        $\zeta$が小さいほど分布は右側にシフトしており孔が成長しやすいと言える.
        (a) $\zeta=20000$.
        (b) $\zeta=16000$.
        (c) $\zeta=12000$.
        (d) $\zeta=8000$.
    }
    \label{fig:result_sphere_friction_constant_comb}
\end{figure}


\subsection{両者のまとめ}
以上でアニーリング速度と粘性抵抗の両者の依存性について述べた.
ここで,本研究で用いた全条件の結果を図\ref{fig:result_sphere_all}に示す.
\begin{figure}
    \centering
    \includesvg[scale=0.9]{image/result_sphere_all.svg}
    \caption{
        本研究で用いた全てのアニーリング速度$t_{max}$と
        粘性抵抗$\zeta$の条件対に対する計算機実験の結果のCG.
    }
\label{fig:result_sphere_all}
\end{figure}
また,その孔の分布を図\ref{fig:result_sphere_all_hist}に示す.
\begin{figure}
    \centering
    \includesvg[scale=0.9]{image/result_sphere_all_hist.svg}
    \caption{
        本研究で用いた全てのアニーリング速度$t_{max}$と
        粘性抵抗$\zeta$の条件対に対する計算機実験の結果の解析結果.
        各ヒストグラムの横軸は正規化された孔の径,縦軸は頻度(8サンプルの合計個数)である.
        右上の数字は関数$y=ae^{-bx}$で回帰したときの$b$の値で,
        この値が大きいほど孔が形成されにくい.
    }
    \label{fig:result_sphere_all_hist}
\end{figure}


\section{考察}
本実験では球面上の粘弾性相分離現象による孔構造のできやすさについて,
アニーリング速度と粘性抵抗の2つの観点から結果を得た.
この節では,
これらの結果について実際のDNAゲルカプセルのパターン形成の実験データと対応させながら考察する.

先行研究ではDNAゲルカプセルのパターンを,
孔のあるヘテロ型と一様なホモ型の2つに分けて議論している.
アニーリング速度とカプセルの大きさに対応する
両型の形成割合に関する統計量が実験により得られており(図\ref{fig:result_moritasan}),
\begin{figure}
    \centering
    \includesvg{image/result_moritasan.svg}
    \caption{
        計算機実験の結果と先行研究の結果の比較.
        % TODO
        % 先行研究の結果\cite{moritasan}.
        % (a) アニーリング速度依存性.
        %     アニーリング速度が速いほど孔の構造は形成されにくい.
        % (b) 液滴の大きさ依存性.
        %     液滴の大きさが大きいほど孔の構造は形成されやすい.
    }
    \label{fig:result_moritasan}
\end{figure}
それらのデータと本計算機実験の解析データとを比較した.


\subsection{アニーリング速度依存性}
先行研究の結果によれば,
アニーリング速度が速いほどホモ型のDNAゲルカプセルが形成されやすく(図\ref{fig:result_moritasan}a),
この結果は数値シミュレーションの結果(図\ref{fig:result_sphere_anearing_speed_comb})と一致する.
系の温度がsticky-endsの融解温度を下回るとY-DNA分子は互いに接続し合い,
完全にゲル化してしまう~\cite{sato2019sequence}.
相分離現象が起こるためには物質の流動が必要であるが,
一度ゲル化してしまうと相分離がそれ以上進まないと考えられる.
したがってアニーリング速度が遅いほど,
つまりゲル化するまでの比較的高温の時間が長いほど孔のパターンは形成されやすいと考えられる.

\subsection{カプセルの半径依存性と粘性依存性の関係}
先行研究のもう一方の結果であるカプセルの大きさの依存性に関しては,
直接カプセルの大きさを変更する数値実験を行っていないため,
本モデルを使い次の仮説を検証する形で考察を行う.

第\ref{sec:dnagel}章で述べたとおりDNAゲルカプセルは球状の液滴の内面に形成されるが,
仮に全てのDNA分子が内面に集積しているとすると,
その総量は球の体積に比例するが,
実際に集積するのは球面(面積)でありその厚さは球の半径に比例することがわかる.
したがって,液滴の大きさが異なればDNAのゲル層の厚さも異なると考えられる.
半径$R$の理想的なDNA溶液の液滴を用いてこれを確認する.
溶液の濃度を$\rho$とし,全てのDNA分子が界面に濃度$\rho'$で均一に集積することを考える.
集積したDNAの層の厚さを$d$とすると,集積の前後で分子の総数は保存されるため
\begin{equation}
    \frac{4}{3}\pi R^3 \rho = \left(\frac{4}{3}\pi R^3-\frac{4}{3}\pi(R-d)^3\right)\rho'
\end{equation}
が成りたつ.
これを$d$について解くと
\begin{equation}
    d = R\left(1-\sqrt[3]{\frac{\rho'-\rho}{\rho'}}\right)
\end{equation}
となり,確かに$d$は$R$に比例することがわかる.

脂質と接するDNA分子は電気的相互作用により脂質膜に吸着するが,
その上に集積してくDNA分子はsticky-endsを介して集合体を形成する.
よって厚く集積するほど脂質膜に吸着しているDNAの割合が小さくなり,
界面上での流動性が高くなる.
以上の議論から,
直径が大きい液滴ほど内面のDNAの流動性は高くなるという仮説をたてた(図~\ref{fig:size_and_friction}).
\begin{figure}
    \centering
    \includesvg{image/size_and_friction.svg}
    \caption{
        DNAゲルカプセル作成時におけるDNA溶液の液滴の大きさと粘性抵抗の関係についての模式図.
        (a) 液滴の直径が大きい場合.
            脂質に近いDNAは電気的な相互作用により引き合うが,
            その上に集積したDNA分子にはほとんど働かず,
            より流動性があると考えられる.
        (b) 液滴の直径が小さい場合.
    }
    \label{fig:size_and_friction}
\end{figure}
ここで言う流動性はモデルにおける粘性抵抗(式\ref{eq:main})と対応すると考えることで,
数値シミュレーション結果における粘性抵抗依存性
(図\ref{fig:result_sphere_friction_constant_comb})
と比較することでこの仮説を肯定的に検証することができる.

先行研究によれば,
液滴の大きさが大きいほどヘテロ型のDNAゲルカプセルが形成されやすく(図\ref{fig:result_moritasan}b), 
この結果は先の仮説を合わせると計算機実験の結果と一致する.


%----------------
% カーンヒリアード
%----------------
\chapter{DNAゲルの特殊なパターン形成}

第\ref{sec:dnagel}章で説明したIn vitroの実験では粘弾性相分離現象と思われる現象をDNAゲル上で再現することに成功したが,
それに付随して別のタイプのパターン形成現象を発見した(図\ref{fig:result_special}a).
この章ではこの現象に対する数理モデルの検討とその計算機実験結果について述べ,
最後に定性的な考察を行う.


\section{実験}
第\ref{sec:dnagel}章の実験方法で説明した方法と同様の方法で作製したバルクDNAを用いる,
このバルクDNAゲルに対し加熱と冷却の温度操作を加える.
用いたステージヒータは第\ref{sec:dnagel}章で用いたものとは異なるステージヒータ
(Tokai-Hit製ThermoPlate)を用いた.
まず\SI{42}{\celsius}まで加熱し,\SI{42}{\celsius}の状態で20から30分放置する.
このインキュベート操作を施すことで,
DNAゲルは第\ref{sec:dnagel}章でみられた形状よりも球に近い形状に近づく.
次に\SI{50}{\celsius}まで加熱し,
ステージヒータのコントローラに表示される温度が\SI{50}{\celsius}に到達した瞬間にステージヒータの電源を切り自然冷却する.

上記の操作の結果,
バルクのDNA集合体は\SI{50}{\celsius}付近では一様に拡散し(図\ref{fig:result_special}b左),
\begin{figure}
    \centering
    \includesvg{image/result_special.svg}
    \caption{
        (a) DNAゲルにみられる特殊なパターン形成現象.
            中央の孔構造は粘弾性相分離現象に依る構造と似ているが,
            外縁部に無数にDNAゲルの粒子が形成される.
        (b) 形成過程の時間経過.
            左の画像はDNAの集合体が拡散している様子.
            中央は冷却時にDNA分子が凝集を始める瞬間の画像.
            右の画像は凝集体が中央と外縁部に明確に分離しゲル化した様子.
    }
    \label{fig:result_special}
\end{figure}
冷却すると拡散したDNA分子が再集合した(図\ref{fig:result_special}b中).
この際に中心に引き戻されるDNAと外縁部で孤立するDNA分子の2つに分けられる.
後者は外縁部で小さなクラスターを形成して冷却と共にゲル化した(図\ref{fig:result_special}b右).


\section{数理モデル}
拡散したDNA分子の周辺部における相分離現象は比較的迅速に引き起こされ,
すぐにDNA分子のクラスターが形成される様子がみられた.
これは中心からの距離に依存し,
中心に近い一定距離内の領域に存在するDNA分子は再び大きな集合体に回復するが,
外側の濃度の低い領域では孤立した集合体を形成する.
粘弾性相分離では途中でネットワーク構造を形成するが,
外側の相分離においてはネットワーク構造の形成が観察されなかったことから,
このパターン形成は通常の相分離現象で説明できると考えた.

そこで,2次元平面上のDNAと水の2成分系を考え,
はじめにDNA分子が拡散を行い,
冷却開始時点を境に相分離が始まるという状況を考えた(図\ref{fig:model_cahn_hilliard}).
\begin{figure}
    \centering
    \includesvg{image/model_cahn_hilliard.svg}
    \caption{
        数理モデルの模式図.
        (a) パターン形成過程の予想の模式図.
            球状のDNAゲルが熱により拡散し,
            その状態で冷却を行うと相分離現象によりパターンが形成される.
        (b) 実際に構築した数理モデル.
            平面上に円状のDNA領域を用意し,
            単純な拡散を行った後にCahn-Hilliardモデルを適用する,
            全体で2段階から成る単純な数理モデルを考える.
    }
    \label{fig:model_cahn_hilliard}
\end{figure}
前半は拡散方程式
\begin{equation}
    \frac{\partial c}{\partial t}
    =
    D \nabla^{2} c
\end{equation}
を使用し,
後半の相分離現象ではスピノーダル分解などを表現する保存系のモデルであるCahn-Hilliardモデル
\begin{equation}
    \frac{\partial c}{\partial t}
    =
    D \nabla^{2} \left( c^3 - c - \gamma \nabla^{2} c \right)
\end{equation}
を用いた.

計算機実験では,間隔が$1$の$256\times 256$の格子を用い,
格子の中心に半径$32$のDNA領域を設けて初期状態とした.
格子の値はDNAの濃度を表現し,$+1$が最大で$-1$が最小である.
時間刻み$\Delta t=0.001$のオイラー法で数値積分し,$t_{mid}=50$,$t_{max}=100$とした(図\ref{fig:model_cahn_hilliard}b参照).


\section{結果}
計算機実験の結果,
$t=t_{max}$にて
DNAゲルの外縁部に形成される無数のDNAゲル粒子の形成を再現することができた
(図\ref{fig:result_cahn_hilliard}).
\begin{figure}
    \centering
    \includesvg{image/result_cahn_hilliard.svg}
    \caption{
        Cahn-Hilliardモデルのパラメータ$D$と$\gamma$を変えて行った実験の結果.
        画像はその条件における計算機実験の$t=t_{max}$の構造.
        構造体の外縁部に無数の小さなDNAの領域が形成されており,
        これはIn vitroの実験で得られた結果と非常に似ている.
    }
    \label{fig:result_cahn_hilliard}
\end{figure}
内部については,実際の構造とは少し異なるパターンが観察された.


\section{考察}
この現象については数値解析が行われていないため定量的な議論ができないが,
以上の結果と前章までの考察を合わせることで可能な考察をここで述べる.

DNA集合体は拡散により同心円状に濃度場を形成する.
中心部分ではヘテロ型のDNAゲル形成の時と同じく,
DNA分子の濃度が十分に高いため通常の粘弾性相分離が引き起こされていると考えられる.
しかし,外縁部では拡散現象によりDNAの濃度が低く分子数が少ないため,
sticky-endsによる弾性の性質を十分に得ることができず,
粘弾性相分離ではなく通常の相分離にとどまっているのではないかと考えた.
その結果,冷却時の外縁部では無数のDNA集合体の形成が直ちに引き起こされ,
このことは数理モデルを用いることでも確認することができた(図\ref{fig:result_cahn_hilliard}).

% 詳細は記述しないが,
% 加熱冷却操作を複数回繰り返すことで内側に階層構造のようなパターンを作ることにも成功しており,
% DNAゲルの新たなパターン形成技術としても興味深い現象であると考える.


%----------------
% 結論
%----------------
\chapter{結論}

自然界には様々なパターン形成現象が存在し,
我々はそれらを数理的に解析することで基礎的な研究に限らず工学的応用も行ってきた.
近年,DNA分子を用いたDNAナノテクノロジーが盛んに研究され,
新しいナノ・マイクロスケールのマテリアルやパターン形成制御法として確立されつつある.
DNA分子の利点はそのプログラマビリティにあり,
他の物質では困難な分子構造や物性などを人間が設計できる可能性を持つ.
本研究ではその中でもY-DNAと呼ばれるDNA分子について注目し,
Y-DNAが作り出す微小なゲルカプセルにみられるパターン形成メカニズムに関して研究を行った.

DNAゲルカプセルには大きく分けて,孔のようなパターンが形成される場合と,
比較的孔が少なく一様なカプセルが形成される場合の2つが存在する.
既に存在する実験結果と今回補助的に行ったIn Vitroの実験の結果などをふまえ,
カプセルにおける孔の形成が粘弾性相分離と呼ばれる現象に起因すると仮定した(第\ref{sec:dnagel}章).
私は,既存の粘弾性相分離の数理モデルを元に,DNAゲルカプセル用の新しい拡張モデルを構築し,
そのパターン形成の数値シミュレーションを行った.
また,そのパターンについての定量的な議論を行うための解析手法についても開発を行い,
両者のコンピュータプログラムの実装を行った.

数値シミュレーションの結果,
実際のDNAゲルカプセルで見られるパターン形成と非常に類似した現象を再現することに成功した.
得られた構造は,形成された孔の大きさの分布を見ることで定量的な解析を行った,
温度や粘性抵抗のパラメータを変え,それらの条件依存性について議論した結果,
先行研究の結果でえられていたパターン形成の傾向についての説明を肯定的に行うことができた.

上記の結果は,DNAゲルカプセルに見られるパターンの形成のメカニズムが
粘弾性相分離現象によるものである可能性を結論付け,
さらにそのパターン制御の可能性についても示唆を与えた.

Y-DNA分子はsticky-endsの長さを変更するとその熱力学的特性の変化から,
Y-DNA集合体全体の物性が変化するという結果が報告されている~\cite{sato2019sequence}.
本研究で扱ったDNAカプセルでは,
温度変化やカプセルの大きさ(粘性抵抗)の観点でパターン形成を議論していたが,
sticky-endsの長さによる形成パターンの変化も十分に考えられる.
したがって,将来的にsticky-endsの長さを変えたDNAゲルカプセル作製の実験が行われることが考えられるが,
今回作成した数理モデルにsticky-endsの長さに対応するパラメータを追加し,
事前にシミュレーションを行うことでIn Vitroの実験をサポートすることも考えられる.

sticky-ends




%----------------
% 謝辞
%----------------
\chapter*{謝辞}

本研究を進めるにあたり瀧ノ上正浩先生には手取り足取りご指導賜りました.
ご多忙にもかかわらずディスカッションの時間を潤沢に用意してくだり,
また設備に困ることもなく非常に恵まれた環境で研究活動を行うことができました.

元・瀧ノ上研究室,現・産業技術総合研究所の森田雅宗博士には貴重な実験データを提供いただきました.
この研究は森田博士の研究なくして成立しませんでした.

佐藤佑介博士にはIn vitroの実験のイロハを教えて頂いただけでなく,
セミナーや日々の何気ない会話の中で研究に関して鋭いアドバイスを下さいました.
私にとって最も身近な先輩の一人で,研究生活をより充実したものにして下さいました.

瀧ノ上研究室に所属する間にお世話になった先輩方・同期・後輩のメンバーは何時も私を受け入れて下さり,
おかげで非常に過ごしやすい研究生活を送ることが叶いました.
また,学生室を共にしていた山村研のメンバーには日常生活でお世話になっただけでなく,
研究に関して様々なアドバイスを頂きました.

秘書の皆様には日々の研究活動のサポートをしていただきました.
無理な注文をお願いしてしまうこともありましたが,
おかげさまで研究活動をスムーズに行うことができました.

研究会等でアドバイスを下さった関連研究者の方々には研究を進める上で様々なヒントをいただきました.
また,研究者がどういった存在でどうあるべきかについてを考えるきっかけになりました.

最後に,いかなる時でも無条件に応援してくださる家族の皆様に感謝申し上げます.
資金面はもちろんのこと,
精神的にも何も杞憂することなく大学で自由に研究生活を送ることができました.

以上に挙げさせていた頂いた方々へ,この場を借りて重ねて感謝を申し上げ,謝辞とさせていただきます.


%----------------
% 参考文献
%----------------
\bibliographystyle{junsrt}
\bibliography{reference}
% \input{appendix}
% ----------------------------------------------------------------------
\end{document}
