\chapter{緒論}

\section{自然界のパターン形成}
私達の身の回りの自然界には非常に多種多様なパターンを見出すことができる.
雪の結晶,山や雲の模様,土や塗料のひび割れなどの見慣れたものから,
植物の葉や枝に見られるフラクタル構造や動物の体表の特徴的な模様など,
生物が自ら作り出すものもある.
パターン形成は目に見えるマクロな世界に限らず,
ミクロな世界でも至るところで行われている.
特に生物の構成要素である細胞では,パターン形成が生命現象に深く関わっていることがわかり始めている.
パターン形成のメカニズムを探ることでその現象の原理や本質を理解できる場合があり,
その行為は基礎科学的にみて重要である.
例えば動物の体表や草本の植生パターンなどにみられる模様は
反応拡散系というモデルにより良く理解されている\cite{turing1990chemical}.

自然界にみられるパターン形成の理解は応用の観点でも興味深いものである.
先に紹介した植物に関する例では,たとえば細胞分裂から着想を得たL-system\cite{lindenmayer1968mathematical}があり,
これは数学的な議論に収まらず,
植物の骨格のフラクタル性と合わせて
植物形態学やコンピュータグラフィクスなどへの応用がなされている\cite{aono1984botanical}.
また,葉脈のフラクタル構造に着目し,そのフラクタル次元を特徴量として種のクラスタリングを行うことで
植物種の同定を葉の葉脈から行う方法が提案されている\cite{bruno2008fractal}.
% TODO 推敲
自然界にみられるこれらのパターン形成は自己組織的に行われることが多く,
工学的な応用を考えた場合にこの特徴はたとえばマテリアルの分野,
特にボトムアップの形成が困難な微小な構造形成に力を発揮する.
スポンジ,特殊表面加工など,例を挙げると枚挙に暇がない.
% TODO 表面張力の本に例があるかもしれない
% TODO もっと例を
%
% さらに


\section{DNAナノテクノロジー}
先に挙げた例はいずれも重要な技術だが,
化学合成などの手法を用いた分子スケールの物質の制御によるパターン制御は未だに困難を極める.
開発には多くの試行錯誤が必要であり時間やコストが無視できない.
人間がプログラマブルなデジタルコンピュータを手に入れたように,
マテリアルの分野においても制御の容易なプログラマブルな物質が求められている.

近年,そのような物質としてDNAが注目されており,
そのナノスケールの工学的応用はDNAナノテクノロジーと呼ばれ盛んに研究が行われている.
DNAは生物の遺伝情報を担う物質として知られているが,
DNAナノテクノロジーではこれをマテリアルとしても利用する.
今日,合成生物学の進歩によりある程度の長さのDNA分子であれば比較的簡単に委託合成が可能で,
合成技術を持たない者でも自分で設計した塩基配列をもつDNA分子の入手が可能である.
さらに,塩基配列をうまく設計すると分岐構造などの構造をつくることができ,
自然界では見られない構造のDNA分子を手に入れることが可能である.
例えば,短い1本鎖DNAを巧みに織り合わせることで多面体の構造を作ったり\cite{chen1991synthesis},
DNAオリガミと呼ばれる規模の大きなDNAナノ構造体を構築する研究が行われている
\cite{rothemund2006folding}.
どちらもナノスケールの構造体で,
従来のボトムアップな工業技術で同様の構造体を得ることは困難である.
設計技術の進歩も著しく,現在ではDNAオリガミ専用のソフトウェアも存在する~\cite{douglas2009rapid}.
設計ソフトウェアの存在でDNAオリガミの応用研究は促進され,
機能を持つさまざまなDNAナノ構造体が報告されている.
また,DNAでつくられたタイルを敷き詰める研究では\cite{winfree1998design},
論理演算を行ったり\cite{mao2000logical},
セル・オートマトンのエミュレーションを行うことで
DNAタイル演算のチューリング万能性を示す研究がある~\cite{rothemund2004algorithmic}.
これらの研究はDNAナノテクノロジーや情報工学に重要な結果を残しただけでなく,
プログラマブルな物質を使用した自己組織的なパターン形成の制御という観点でも最も成功した例の1つある.
これらの研究例からわかるように,
DNAは先に説明したプログラマブルな物質としての条件を満たしていると考えることができる.
% TODO DNAの説明を詳しく

本研究ではDNAナノテクノロジーの応用の1つのDNAゲルを扱う.
DNAゲルは分岐構造を持つDNA分子を用いてゲルを形成させる技術で~\cite{um2006enzyme},
本研究室の先行研究では,このゲルを用いたマイクロスケールの構造体の作製に成功している.
主に人工細胞の要素技術への応用が考えられており,
その1つは細胞をみたてた脂質二重膜でできたマイクロスケールの小胞の内側にDNAゲルを裏打ちさせる技術である.
DNAゲルの裏打ち構造により,
この小胞は過酷な浸透圧変化に耐えられることが知られている\cite{kurokawa2017dna}.
また,DNAゲル自身によるカプセル構造(DNAゲルカプセル)の作製も行われている
\cite{morita2017formation}.
このカプセル作製に遺伝情報をコーディングしたDNAゲルを使用し,
そのゲルの収縮を利用してRAN転写効率を制御する方法が提案されている~\cite{watanabe}.
これらのDNAゲルによるマイクロ構造体はその生体親和性からドラッグデリバリーへの応用も期待される.

DNAゲルカプセルの作製においても,自己組織的なパターン形成が応用されている.
1つはゲルを構成するDNA分子形成であるナノスケールのパターン形成で,
もう1つはゲルのマクロな形状を決定するパターン形成である.
前者は熱力学を用いた予測や制御が可能になりつつあるが~\cite{zadeh2011nupack}, 
後者についてはメカニズムが明確に理解されない場合が多い.


\section{研究目的}
本研究では特にDNAゲルカプセルにみられる孔状のパターン形成に注目し,
そのメカニズムを主に数値解析により明らかにすることである.
そのために,
DNAゲルカプセルに見られるパターン形成は粘弾性相分離現象というモデルで説明できると仮定し,
カプセルを見立てた球面上に数理モデルを拡張し計算機実験を行った.
また,計算機実験の結果を先行研究のIn vitroの実験結果と定量的に比較するための解析方法についても開発を行った.
%In vitro の実験を行う上で予測をたてることに


\section{論文構成}
本論文は全6章から構成される.
第1章では,
緒論として自然界のパターン形成のその工学的有用性,
またDNAナノテクノロジーと本研究との関連について述べた.
第2章では,
本研究で扱うDNAゲルカプセルについての詳細を述べる.
この内容は主に本研究室で行われたMoritaらの先行研究に基づく~\cite{morita2017formation}.
また,
パターン形成が脂質に依らないことを裏付けるために行ったIn vitroの補助実験について述べ,
メカニズムが粘弾性相分離現象とよばれるモデルで説明できる可能性について説明する.
第3章では,
計算機実験の手法について述べる.
初めに数理モデルの説明を行い,
私が施した拡張ついて述べる.
次に計算機実験の結果を定量的に解析するために開発した手法について説明する.
第4章では,
計算機実験の結果を述べ,
先行研究の結果との比較を行い考察する.
第5章では,
第2章で述べたIn vitroの実験時に付随して発見したDNAゲルにおける特殊なパターン形成について述べる.
最後に第6章で結論をし総括する.
