\chapter*{概要}

自然界には自己組織的なパターン形成現象が多種多様に存在し,
我々はそれらを数理的に理解し工学などに応用してきた.
近年,プログラマブルな分子としてDNAが注目され,
DNAナノテクノロジーと呼ばれる分野で研究が盛んに行われている.
DNAを用いたパターン形成の研究も行われており,
本研究ではDNAで作られる微小なゲルカプセル(DNAゲルカプセル)上のパターン形成を扱う.
DNAゲルカプセルには大きく2種類のパターンが見られることが先行研究により知られており,
両者を分ける条件やパターン形成メカニズムを解明することが本研究の目的である.
私が補助的に行ったIn vitroの実験結果や先行研究の結果をふまえ,
そのパターン形成が粘弾性相分離と呼ばれる現象を用いて説明できると考えた.
そこで,粘弾性相分離現象の数理モデルを球面上に拡張し,
DNAゲルカプセルをみたてたパターン形成の計算機実験を行った.
また,得られた結果を定量的に解析する手法についても新たに考案した.
実験の結果,実際のDNAゲルカプセルに類似した構造を得ることに成功し,
その解析結果も先行研究で得られていた結果と一致した.
この結果により,
従来予想されていたDNAゲルカプセルのパターン形成メカニズムの仮説を肯定的にうことができた.
