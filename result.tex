\chapter{結果}

数値シミュレーションの結果を述べる.
まず,既存のArakiらのモデルの再現実験を行い,
温度変化の拡張,
球面上への拡張の順にその結果を述べ,
先行研究のDNAゲルカプセルとの比較を行う.

\section{平面上の温度変化なしの結果}

まず,Arakiらのモデルの再現実験の結果を述べる.
%TODO

\section{平面上の温度変化ありの結果}

次に,温度変化の拡張を行った結果について述べる.


\section{球面上の温度変化ありの結果}

次に,モデルを球面上に拡張した場合の結果を述べる.
%TODO 粘性固定

\subsection{アニーリング速度依存性}


\subsection{カプセルの半径依存性}
先行研究では,カプセルの大きさに依存して孔の形成されやすさが変わることも報告されている~\cite{morita2017formation}.
そこで,その理由について本モデルを使い次の仮説を検証した.

第\ref{sec:dnagel}章で述べたとおりDNAゲルカプセルは球状の液滴の内面に形成されるが,
仮に全てのDNA分子が内面に集積しているとすると,
その総量は球の体積に比例するが,
実際に集積するのは球面(面積)であり,
その厚さは球の半径に比例することがわかる.
したがって,液滴の大きさにより集積したDNAの厚さは異なると考えられる.
より厚く集積するほど内側のDNAは脂質膜から遠くなるため,
脂質との電気的な相互作用が弱くなり流動性が高くなる.
つまり,直径が大きい液滴ほど内面のDNAの流動性は高くなると仮説をたてた.

ここで言う流動性はモデルの粘性抵抗(式\ref{eq:main})と対応すると考え,
その値を変えてこの仮説を検証する実験を行った.
%TODO 図
