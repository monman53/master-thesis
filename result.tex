\chapter{結果}
\label{sec:result}

第\ref{sec:model}章で説明した数値シミュレーションの実験結果とその解析結果について述べる.
平面上のモデルと球面上のモデルについて,
それぞれ温度変化の有無による計算結果を述べ,
最後に先行研究のDNAゲルカプセルとの比較し考察を行う.
%TODO 球面の影響についても言及?

\section{平面上モデルの結果}

\subsection{温度変化なし}

まず,今回施した拡張を一切適用しないオリジナルのモデルを用いたシミュレーション結果について述べる.
計算点は縦40,横40の計1600点からなり,長さ$1$のバネで接続された状態から計算を開始する.
各定数については
$\zeta=12000$,$\epsilon=0.4$,$\sigma=0.25$,$\Delta t=1$,$T=0.45, E_0=6.5$の条件で実験を行った.

前半では全体に孔構造が形成されており,
この孔の部分には拡散の速い物質が集合しているとみなすことができる(図\ref{fig:result_2d_without_anearing}b).
時間が経過するにつれ形成された孔が成長し(図\ref{fig:result_2d_without_anearing}c),
系全体でネットワーク構造が形成される(図\ref{fig:result_2d_without_anearing}d).
ネットワーク構造を形成していた粒子は次第にまとまりはじめ(図\ref{fig:result_2d_without_anearing}e-g),
最終的には前半で見られた孔のパターンとは成分が完全に逆転し,
粒子が集合しクラスターを構成する状態で安定となる(図\ref{fig:result_2d_without_anearing}h).
これは粘弾性相分離現象の一連の過程と類似しており,元のモデルが表現していた現象である.

\begin{figure}
\centering
\includesvg{image/result_2d_without_anearing.svg}
\caption{
    拡張を施す前の平面モデルの計算結果の時間変化.
    (a) $t=0$.
    (b) $t=10000$.
    (c) $t=20000$.
    (d) $t=30000$.
    (e) $t=40000$.
    (f) $t=50000$.
    (g) $t=60000$.
    (h) $t=70000$.
}
\label{fig:result_2d_without_anearing}
\end{figure}

\subsection{温度変化あり}

次に,温度変化の拡張を行った結果について述べる.
温度$T$以外の条件を先の温度変化なしの実験と同じにし,
温度$T$を式\ref{eq:thermal}のように変更し対照実験を行う.
ただし$T_{max}=0.45$,$t_{max}=80000$とし,
この値は先の実験における$T$やシミュレーション時間と同じであるので,
両者の異なる点は$T$の温度が線形に小さくなっていく点のみである.

前半は温度変化のない場合と同じように全体に孔の構造が形成される(図\ref{fig:result_2d_with_anearing}a-d).
後半はその孔が成長するものの,ネットワーク構造を保ったまま安定状態になる(図\ref{fig:result_2d_with_anearing}e-h).
これは温度が下がることにより揺動力やバネの切断確率が小さくなることに起因する.

実際のDNAカプセルゲルも,
通常の粘弾性相分離のように構成成分がクラスターを形成するまで相分離が進行せず,
この実験で得られた結果のように途中のネットワーク構造を形成する段階でゲル化し安定化していると考えられる.

\begin{figure}
\centering
\includesvg{image/result_2d_with_anearing.svg}
\caption{
    温度変化の拡張を施した平面モデルの計算結果の時間変化.
    (a) $t=0$.
    (b) $t=10000$.
    (c) $t=20000$.
    (d) $t=30000$.
    (e) $t=40000$.
    (f) $t=50000$.
    (g) $t=60000$.
    (h) $t=70000$.
}
\label{fig:result_2d_with_anearing}
\end{figure}

\subsubsection{アニーリング速度依存性}
アニーリング速度を変えた場合のパターンの違いについて述べる.
アニーリング速度を変えて実験するためには,
式\ref{eq:thermal}より$t_{max}$を変更すれば良い.
$t_{max}$が小さいほどアニーリング速度が速いことになる.

アニーリング速度以外の条件は先の実験と同じに設定し,
$t_{max}=20000$,$t_{max}=40000$,$t_{max}=60000$,$t_{max}=80000$の場合について実験を行った(図\ref{fig:result_2d_anearing_speed}).
実験の結果,アニーリング速度により構造に大きな違いが現れることが確認された.
アニーリング速度が大きいほど孔は形成されにくい傾向にあり,
形成されたとしても非常に小さいことが確認できる(図\ref{fig:result_2d_anearing_speed}(a)).
これはバネの切れやすい温度の高い状態にある時間が少ないためであると考えられる.

孔の大きさの分布などの定量的な解析については,
球面上に拡張したモデルにおける実験結果において議論する.
%TODO 解析結果ももし可能であれば載せたい

\begin{figure}
\centering
\includesvg{image/result_2d_anearing_speed.svg}
\caption{
    平面モデルにおける構造のアニーリング速度依存性.
    温度変化ありのモデルにおいてアニーリング速度を変えて実験するためには$t_{max}$を変更する.
    式\ref{eq:thermal}より,$t_{max}$が小さいほどアニーリング速度が速い.
    画像は$t=t_{max}$のときの構造のCGである.
    (a) $t_{max}=20000$.
    (b) $t_{max}=40000$.
    (c) $t_{max}=60000$.
    (d) $t_{max}=80000$.
}
\label{fig:result_2d_anearing_speed}
\end{figure}




\section{球面上モデルの結果}

次に,モデルを球面上に拡張した場合の結果を述べる.

\subsection{温度変化なし}

まず,粒子数以外を平面モデルと同じ条件にし,
温度変化を行わない条件で実験を行った.

\begin{figure}
\centering
\includesvg{image/result_sphere_without_anearing.svg}
\caption{
    温度変化の無い球面モデルの計算結果の時間変化.
    (a) $t=0$.
    (b) $t=10000$.
    (c) $t=20000$.
    (d) $t=30000$.
    (e) $t=40000$.
    (f) $t=50000$.
    (g) $t=60000$.
    (h) $t=70000$.
}
\label{fig:result_sphere_without_anearing}
\end{figure}


\subsection{温度変化あり}

温度変化ありの場合のシミュレーション結果を図\ref{fig:result_sphere_with_anearing}に示す.
平面モデルと同じように,
初めに形成された小さな孔が次第に成長する様子が確認された.
最終状態のパターンに関して,
実験条件を変えて解析を行った.

\begin{figure}
\centering
\includesvg{image/result_sphere_with_anearing.svg}
\caption{
    温度変化の拡張を施した球面モデルの計算結果の時間変化.
    (a) $t=0$.
    (b) $t=10000$.
    (c) $t=20000$.
    (d) $t=30000$.
    (e) $t=40000$.
    (f) $t=50000$.
    (g) $t=60000$.
    (h) $t=70000$.
}
\label{fig:result_sphere_with_anearing}
\end{figure}

\subsubsection{アニーリング速度依存性}

粘性抵抗$\zeta$の値を固定し,
$t_{max}$の値を変えて実験を行った\ref{fig:result_sphere_anearing_speed}.

\begin{figure}
\centering
\includesvg{image/result_sphere_anearing_speed.svg}
\caption{
    球面モデルにおける構造のアニーリング速度依存性.
    温度変化ありのモデルにおいてアニーリング速度を変えて実験するためには$t_{max}$を変更する.
    式\ref{eq:thermal}より,$t_{max}$が小さいほどアニーリング速度が速い.
    画像は,$t=t_{max}$のときの構造のCGである.
    (a) $t_{max}=20000$.
    (b) $t_{max}=40000$.
    (c) $t_{max}=60000$.
    (d) $t_{max}=80000$.
}
\label{fig:result_sphere_anearing_speed}
\end{figure}

$t=t_{max}$の構造の解析を行ったところ,
アニーリング速度が大きいほど($t_{max}$が小さいほど)孔の形成がされにくいことがわかった
\ref{fig:result_sphere_anearing_speed_hist}.

\begin{figure}
\centering
\includesvg{image/result_sphere_anearing_speed_hist.svg}
\caption{
    孔の大きさの頻度分布のあニーリング速度$\zeta$依存性.
    図\ref{fig:result_sphere_anearing_speed}で示した構造に対する解析結果.
    アニーリング速度が遅いほど($t_{max}$が大きいほど)
    分布は右側にシフトしており孔が成長しやすいと言える.
    (a) $t_{max}=20000$.
    (b) $t_{max}=40000$.
    (c) $t_{max}=60000$.
    (d) $t_{max}=80000$.
}
\label{fig:result_sphere_anearing_speed_hist}
\end{figure}

\subsubsection{粘性抵抗依存性}

次に$t_{max}$の値を固定し,
粘性抵抗$\zeta$の値を変えて実験を行った
\ref{fig:result_sphere_friction_constant}

\begin{figure}
\centering
\includesvg{image/result_sphere_friction_constant.svg}
\caption{
    球面モデルにおける構造の粘性抵抗依存性.
    (a) $\zeta=8000$.
    (b) $\zeta=12000$.
    (c) $\zeta=16000$.
    (d) $\zeta=20000$.
}
\label{fig:result_sphere_friction_constant}
\end{figure}

$t=t_{max}$の構造の解析を行ったところ,
粘性抵抗$\zeta$が大きいほど孔の形成がされにくいことがわかった
\ref{fig:result_sphere_friction_constant}.

\begin{figure}
\centering
\includesvg{image/result_sphere_friction_constant_hist.svg}
\caption{
    孔の大きさの頻度分布の粘性抵抗$\zeta$依存性.
    図\ref{fig:result_sphere_friction_constant}で示した構造に対する解析結果.
    $\zeta$が小さいほど分布は右側にシフトしており孔が成長しやすいと言える.
    (a) $\zeta=8000$.
    (b) $\zeta=12000$.
    (c) $\zeta=16000$.
    (d) $\zeta=20000$.
}
\label{fig:result_sphere_friction_constant_hist}
\end{figure}


\section{考察}

%TODO 先行研究の結果の図

\subsubsection{カプセルの半径依存性}

先行研究では,カプセルの大きさに依存して孔の形成されやすさが変わることも報告されている~\cite{morita2017formation}.
そこで,その理由について本モデルを使い次の仮説を検証した.

第\ref{sec:dnagel}章で述べたとおりDNAゲルカプセルは球状の液滴の内面に形成されるが,
仮に全てのDNA分子が内面に集積しているとすると,
その総量は球の体積に比例するが,
実際に集積するのは球面(面積)であり,
その厚さは球の半径に比例することがわかる.
したがって,液滴の大きさにより集積したDNAの厚さは異なると考えられる.
より厚く集積するほど内側のDNAは脂質膜から遠くなるため,
脂質との電気的な相互作用が弱くなり流動性が高くなる.
つまり,直径が大きい液滴ほど内面のDNAの流動性は高くなると仮説をたてた.
ここで言う流動性はモデルの粘性抵抗(式\ref{eq:main})と対応すると考え,
その値を変えてこの仮説を検証する実験を行った.
%TODO 図
