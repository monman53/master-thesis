\chapter*{概要}

自然界には自己組織的なパターン形成現象が多種多様に存在し,
我々はそれらを数理的に理解し工学などに応用してきた.
しかし,物性やパターンの制御は一般に困難であり,
デジタルコンピュータのようなプログラマビリティをマテリアルの分野で実現するのは容易くない.
一方で近年,プログラマブルな分子としてDNAが注目され,
DNAナノテクノロジーと呼ばれる分野で研究が盛んに行われている.
DNAを用いてプログラマブルにパターン形成を制御する研究も行われており,
本研究でもDNAで作られる微小なゲルカプセル(DNAゲルカプセル)上のパターン形成を扱う.
%%%%%%%%%
DNAゲルカプセルには孔の開いたヘテロ型と,
孔の無い一様なホモ型の大きく分けて2種類のパターンがあることが先行研究により知られている.
両者の違いを決定する実験条件を解明することや,
パターン形成メカニズムを理解することが本研究の目的である.
私が初めに補助的に行ったIn vitroの実験や先行研究の結果をふまえ,
そのパターン形成が粘弾性相分離と呼ばれる現象を用いて説明できると考えた.
そこで,粘弾性相分離現象の数理モデルを球面上に拡張し,
DNAゲルカプセルをみたてたパターン形成の計算機実験を行った.
数理モデルを球面上に拡張する方法は今回新たに開発し,
得られた結果を定量的に解析する手法も新たに考案した.
%
実験の結果,実際のDNAゲルカプセルに類似した構造を得ることに成功し,
その定量的な解析結果も先行研究で得られていたパターン形成の傾向の実験データと一致した.
具体的には,カプセル作製時のアニーリング速度とカプセルの大きさに注目し,
それぞれに対応する数理モデルのパラメータを変えて計算機実験を行った.
その結果,アニーリング速度が速いほどホモ型になりやすく,
カプセルの大きさが大きいほどヘテロ型になりやすいことが分かった.
%
本研究の結果により,
従来予想されていたDNAゲルカプセルのパターン形成メカニズムの仮説を肯定的に説明することができた.
提案した計算機実験は,
今後のDNAゲルカプセルのIn vitroの実験をサポートすることが期待できる.
また数理モデルについて,
塩基配列に由来する離散的なパラメータを用いたDNA特有の新しい数値計算の手法の開発を展望として挙げる.
