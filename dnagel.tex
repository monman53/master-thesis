\chapter{DNAゲルカプセル}

\section{DNAゲル}

本研究で扱うDNAゲルは,Y-DNAと呼ばれる人工的に合成したDNA分子を用いる~\cite{sato2019sequence}.
Y-DNA分子は3本の1本鎖DNAがY字形状にハイブリダイゼーションした構造をとる(図\ref{fig:ydna}b).
それぞれの1本鎖DNA分子の塩基配列は,
Y字形状をとるように事前にコンピュータソフトウェア~\cite{zadeh2011nupack}を用いて設計される.
3本の腕の先端は完全に相補鎖を形成せず,1本鎖の状態で突出している.
この1本鎖の部分の塩基配列は回文配列になっており,
他の分子の腕の先とハイブリダイゼーションしうる状態になっている.
我々はこの部分を``sticky-ends''と呼んでいる.
%TODO 図を入れる
Y-DNA分子同士がsticky-endを介して相互作用しネットワークを形成することで,ゲルが形成される(図\ref{fig:ydna}c).

\begin{figure}
\centering
\includesvg{image/ydna.svg}
\caption{Y-DNAの模式図.
    (a) Y-DNA分子を構成する3種類の一本鎖DNAの塩基配列.
        赤色の部分はsticky-endsと呼ばれる回文配列で,sticky-ends同士は相補の配列である.
    (b) アニーリング操作により3本の一本鎖DNAはY字形状にハイブリダイゼーションする.
    (c) 溶液中でY-DNA分子はsticky-endsを介してネットワークを形成し,ゲル状態になる.
}

\label{fig:ydna}
\end{figure}

\section{DNAゲルカプセル}

DNAゲルをカプセル状に作製したものがDNAゲルカプセルである.
おおまかな作製方法は図~\ref{fig:capsule}の見出しに記述したが,
詳細については次に述べる.

用意するものは液相と油相の2種類の液体である.
油相はミネラルオイルと脂質の混合物で,
脂質はDOPCとDOTAPの二種類から構成される.
% DOPCは帯電していないが,DOTAPは正に帯電している.
液相はY-DNAを構成する1本鎖DNAを水に溶かしたもので,
緩衝剤やナトリウムイオン,DNAの蛍光用色素なども含まれる.

両者を容器に入れ加熱するとY-DNA分子は完全に1本鎖の状態になる.
次に,容器をタッピングし撹拌することでマイクロスケールの液滴を油中に作製する.
液滴と油相の界面には脂質が一重膜を形成しているが,
このときDOTAPは正に帯電しているため膜は正電荷をもつ(DOTAPは帯電していない).
一方でDNA分子は一般に負に帯電しているため,電気的性質により両者は引き合う.
その結果,DNA分子は液滴内面に集積する.
この状態でアニーリング操作を加えることで3種類の1本鎖DNA分子はY-DNA分子を形成し,
しだいにゲル化する.

このゲル化の過程で形成されるパターンには2種類あることが先行研究により報告されている~\cite{morita2017formation}.
%TODO 推敲
実験条件による傾向も確認されているが,
両者を分ける明確な理由は明らかになっていない.

\begin{figure}
\centering
\includesvg{image/capsule.svg}
\caption{DNAゲルカプセルの作製方法.
    (a) 脂質を混ぜた油とY-DNA溶液を容器に入れ加熱する.
        加熱後,容器を撹拌しマイクロスケールの微小な液滴を作製する.
        この状態でアニーリングを施すことにより,
        液滴内面にDNAゲルが形成されカプセルとなる.
    (b) DNAゲルカプセルが液滴内部で形成される過程の詳細.
        一般に負に帯電しているDNA分子は,正に帯電している脂質膜上に集積する.
        この状態でアニーリング操作を加えることでDNA分子はカプセル状にゲル化する.
        この過程で一様な形状とそうではない形状の2種類のカプセルが形成されることがわかっている.
        この2種類の形成メカニズムを解明することが本研究の主な目的である.
        一番右の画像は実際のDNAゲルカプセルの顕微鏡による蛍光観察の画像で,
        Moritaらの研究結果から拝借した~\cite{morita2017formation}.
}

\label{fig:capsule}
\end{figure}
