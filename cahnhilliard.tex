\chapter{DNAゲルの特殊なパターン形成}

第\ref{sec:dnagel}章で説明したIn vitroの実験では粘弾性相分離現象と思われる現象をDNAゲル上で再現することに成功したが,
それに付随して別のタイプのパターン形成現象を発見した(図\ref{fig:result_special}a).
この章ではこの現象に対する数理モデルの検討とその計算機実験結果について述べ,
最後に定性的な考察を行う.

\section{実験}
第\ref{sec:dnagel}章の実験方法で説明した方法と同様の方法で作製したバルクDNAを用いる,
このバルクDNAゲルに対し加熱と冷却の温度操作を加える.
用いたステージヒータは第\ref{sec:dnagel}章で用いたものとは異なるステージヒータ
(Tokai-Hit製ThermoPlate)を用いた.
まず\SI{42}{\celsius}まで加熱し,\SI{42}{\celsius}の状態で20から30分放置する.
このインキュベート操作を施すことで,
DNAゲルは第\ref{sec:dnagel}章でみられた形状よりも球に近い形状に近づく.
次に\SI{50}{\celsius}まで加熱し,
ステージヒータのコントローラに表示される温度が\SI{50}{\celsius}に到達した瞬間にステージヒータの電源を切り自然冷却する.

上記の操作の結果,
バルクのDNA集合体は\SI{50}{\celsius}付近では一様に拡散し(図\ref{fig:result_special}b左),
冷却すると拡散したDNA分子が再集合した(図\ref{fig:result_special}b中).
この際に中心に引き戻されるDNAと外縁部で孤立するDNA分子の2つに分けられる.
後者は外縁部で小さなクラスターを形成して冷却と共にゲル化した(図\ref{fig:result_special}b右).

\begin{figure}
\centering
\includesvg{image/result_special.svg}
\caption{
    (a) DNAゲルにみられる特殊なパターン形成現象.
        中央の孔構造は粘弾性相分離現象に依る構造と似ているが,
        外縁部に無数にDNAゲルの粒子が形成される.
    (b) 形成過程の時間経過.
        左の画像はDNAの集合体が拡散している様子.
        中央は冷却時にDNA分子が凝集を始める瞬間の画像.
        右の画像は凝集体が中央と外縁部に明確に分離しゲル化した様子.
}
\label{fig:result_special}
\end{figure}

\section{数理モデル}
拡散したDNA分子の周辺部における相分離現象は比較的迅速に引き起こされ,
すぐにDNA分子のクラスターが形成される様子がみられた.
これは中心からの距離に依存し,
中心に近い一定距離内の領域に存在するDNA分子は再び大きな集合体に回復するが,
外側の濃度の小さい領域では孤立した集合体を形成する.
粘弾性相分離では途中でネットワーク構造を形成するが,
外側の相分離においてはネットワーク構造の形成が観察されなかったことから,
このパターン形成は通常の相分離現象で説明できると考えた.

そこで,2次元のDNAと水の2成分系を考え,
はじめにDNA分子が拡散を行い,
冷却開始時点を境に相分離が始まるという状況を考えた(図\ref{fig:model_cahn_hilliard}).
前半は次の拡散方程式を使用し,

\begin{figure}
\centering
\includesvg{image/model_cahn_hilliard.svg}
\caption{
    数理モデルの模式図.
    (a) パターン形成過程の予想の模式図.
    (b) 実際に構築した数理モデル.
        平面上に円状のDNA領域を用意し,
        単純な拡散を行った後にCahn-Hilliardモデルを適用する,
        全体で2段階から成る単純な数理モデルを考える.
}
\label{fig:model_cahn_hilliard}
\end{figure}

\begin{equation}
\frac{\partial c}{\partial t}
=
D \nabla^{2} c
,
\end{equation}

後半の相分離現象ではスピノーダル分解などを表現する保存系のモデルであるCahn-Hilliardモデルを用いた.

\begin{equation}
\frac{\partial c}{\partial t}
=
D \nabla^{2} \left( c^3 - c - \gamma \nabla^{2} c \right)
.
\end{equation}

実験では間隔が$1$の$256\times 256$の格子を用い,中心に半径32のDNA領域を設けて初期状態とした.
格子の値はDNAの濃度を表現し,$+1$が最大で$-1$が最小である.
時間刻み$dt=0.001$のオイラー法で数値積分し,$t_{mid}=50$,$t_{max}=100$とする(図\ref{fig:model_cahn_hilliard}b参照).

\section{結果}

計算機実験の結果,
$t=t_{max}$にて
DNAゲルの外縁部に形成される無数のDNAゲル粒子の形成を再現することができた
(図\ref{fig:result_cahn_hilliard}).
内部については,実際の構造とは少し異なるパターンが観察された.

\begin{figure}
\centering
\includesvg{image/result_cahn_hilliard.svg}
\caption{
    Cahn-Hilliardモデルのパラメータ$D$と$\gamma$を変えて行った実験の結果.
    画像はその条件におけるシミュレーションの$t=t_{max}$の構造.
    構造体の外縁部に無数の小さなDNAの領域が形成されており,
    これはIn vitroの実験で得られた結果と非常に似ている.
}
\label{fig:result_cahn_hilliard}
\end{figure}


\section{考察}
この現象については数値解析が行われていないため定量的な議論ができないが,
以上の結果と前章までの考察を合わせることで可能な考察をここで述べる.

DNA集合体は拡散により同心円状に濃度場を形成する.
中心部分ではヘテロ型のDNAゲル形成の時と同じく,
DNA分子の濃度が十分に大きいため通常の粘弾性相分離が引き起こされていると考えられる.
しかし,外縁部では拡散現象によりDNAの濃度が小さく分子数が少ないため,
sticky-endsによる弾性の性質を十分に得ることができず,
粘弾性相分離ではなく通常の相分離にとどまっているのではないかと考えた.
その結果,冷却時の外縁部では無数のDNA集合体の形成が直ちに引き起こされ,
このことは数理モデルを用いることでも確認することができた(図\ref{fig:result_cahn_hilliard}).

% 詳細は記述しないが,
% 加熱冷却操作を複数回繰り返すことで内側に階層構造のようなパターンを作ることにも成功しており,
% DNAゲルの新たなパターン形成技術としても興味深い現象であると考える.
