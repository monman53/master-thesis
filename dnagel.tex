\chapter{DNAゲルカプセル}

\section{DNAゲル}

本研究で扱うDNAゲルは,Y-DNAと呼ばれる人工的に合成したDNA分子を用いる~\cite{sato2019sequence}.
Y-DNA分子は3本の1本鎖DNAがY字形状にハイブリダイゼーションした構造をとる.
それぞれの1本鎖DNA分子の塩基配列は,
Y字形状をとるように事前にコンピュータソフトウェア~\cite{zadeh2011nupack}を用いて設計される.
3本の腕の先端は完全に相補鎖を形成せず,1本鎖の状態で突出している.
この1本鎖の部分の塩基配列は回文配列になっており,
他の分子の腕の先とハイブリダイゼーションしうる状態になっている.
我々はこの部分を``sticky-ends''と呼んでいる.
%TODO 図を入れる
Y-DNA分子はsticky-endを介して相互作用し,
分子同士のネットワークを形成することでゲルを形成すると考えられている.

\section{DNAゲルカプセル}

DNAゲルをカプセル状に形成したものがDNAゲルカプセルである.

