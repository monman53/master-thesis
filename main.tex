% \documentclass[12pt, a4paper]{jsreport}
\documentclass[a4paper,11pt,oneside,openany,report]{jsbook}
% \usepackage[left=30mm,right=30mm,top=25mm,bottom=25mm]{geometry}
%================
% パッケージ・マクロ
%================
% テンプレート
\usepackage{cscover}

% for hyperref
\usepackage[a4paper,dvipdfmx,pdfdisplaydoctitle=true,%
    bookmarks=true,bookmarksnumbered=true,bookmarkstype=toc,bookmarksopen=true,%
    pdftitle={DNAゲルのミクロパターン形成の数理モデルの構築と数値的な解析},%
    pdfauthor={阪本哲郎}%
    ]{hyperref}
\usepackage{pxjahyper}
\hypersetup{
    % pdfborder = {0 0 0},
}
% フォント
\AtBeginDvi{\special{pdf:mapfile ptex-ipa.map}}
% 数学関連
\usepackage{amsmath,amsthm,amssymb,ascmac}
\usepackage{bm}
% 画像
\usepackage{svg}
% 単位
\usepackage{siunitx}
% 引用数字を行頭にしない
\usepackage[nobreak]{cite}
% ヘッダ
% \pagestyle{headings}



\renewcommand{\bibname}{参考文献}
\setcounter{tocdepth}{2}
\pagestyle{plain}

\newcommand{\TODO}[1]{\textbf{[TODO: #1]}}
%\renewcommand{\TODO}[1]{}

%================
% メタ情報
%================

\thesistype{修士論文}
%\thesistype{博士論文}
% \title{学位論文の\\体裁に関する研究}
\title{DNAゲルのミクロパターン形成の\\数理モデルの構築と数値的な解析}
\author{阪本 哲郎}
\studentid{18M31094}
\date{2020-01-27}
\affiliation{%
  東京工業大学\\
  情報理工学院\\
  知能情報コース
  % 情報工学コース
}

\supervisorname{指導教員}
\supervisor{瀧ノ上 正浩}
% \dsupervisorname{副査教員}
% \dsupervisor{山村 雅幸}
% \dsupervisorname{副査教員}
% \dsupervisor{関嶋 政和}



\begin{document}

%================
% 表紙
%================
\frontmatter
\maketitle

%================
% 概要
%================
\chapter*{概要}

概要

\clearpage

%================
% 目次
%================
\setcounter{tocdepth}{2}
\tableofcontents
% \listoffigures


%================
% 内容
%================
\mainmatter

%----------------
% イントロ
%----------------
\chapter{緒論}

\section{自然界のパターン形成}
私達の身の回りの自然界には非常に多種多様なパターンを見出すことができる.
雪の結晶,山や雲の模様,土や塗料のひび割れなどの見慣れたものから,
植物の葉や枝に見られるフラクタル構造や動物の体表の特徴的な模様など,
生物が自ら作り出すものもある.
パターン形成は目に見えるマクロな世界に限らず,
ミクロな世界でも至るところで行われている.
特に生物の構成要素である細胞では,パターン形成が生命現象に深く関わっていることがわかり始めている.
パターン形成のメカニズムを探ることでその現象の原理や本質を理解できる場合があり,
その行為は基礎科学的にみて重要である.
例えば動物の体表や草本の植生パターンなどにみられる模様は反応拡散系というモデルにより良く理解されている~\cite{turing1990chemical}.

自然界にみられるパターン形成の理解は応用の観点でも興味深いものである.
先に紹介した植物に関する例では,たとえば細胞分裂から着想を得たL-system~\cite{lindenmayer1968mathematical}があり,
これは数学的な議論に収まらず植物の骨格のフラクタル性と合わせて,
植物形態学やコンピュータグラフィクスなどへの応用がなされている~\cite{aono1984botanical}.
また,葉脈のフラクタル構造に着目し,そのフラクタル次元を特徴量として種のクラスタリングを行うことで
植物種の同定を葉の葉脈から行う方法が提案されている~\cite{bruno2008fractal}.
% TODO 推敲
これらの自然界にみられるパターン形成は自己組織的に行われることが多く,
この特徴はマテリアルの分野,
特にボトムアップの形成が困難な微小な構造形成に力を発揮する.
スポンジ,特殊表面加工など,例を挙げると枚挙に暇がない.
% TODO 表面張力の本に例があるかもしれない
% TODO もっと例を
%
% さらに


\section{DNAナノテクノロジー}
先に挙げた例はいずれも重要な技術だが,
化学合成などの手法を用いた分子スケールの物質の制御は未だに困難を極める.
開発には多くの試行錯誤が必要であり時間やコストが無視できない.
人間がプログラマブルなデジタルコンピュータを手に入れたように,
マテリアルの分野においても制御の容易なプログラマブルな物質が求められている.

近年,そのような物質としてDNAが注目されており,
そのナノスケールの工学的応用はDNAナノテクノロジーと呼ばれ盛んに研究が行われている.
DNAは生物の遺伝情報を担う物質として知られているが,
DNAナノテクノロジーではこれをマテリアルとしても利用する.
今日,合成生物学の進歩によりある程度の長さのDNA分子であれば比較的簡単に委託合成が可能で,
合成技術を持たない者でも自分で設計した塩基をもつDNAの入手が可能である.
塩基配列をうまく設計すると分岐構造などの構造をつくることができ,
自然界では見られない構造のDNA分子を手に入れることが可能である.
例えば,短い1本鎖DNAを巧みに織り合わせることで多面体の構造を作ったり~\cite{chen1991synthesis},
DNAオリガミと呼ばれる規模の大きなDNAナノ構造体の研究が行われている~\cite{rothemund2006folding}.
どちらもナノスケールの構造体で,従来のボトムアップな工業技術での実現は困難である.
設計技術の進歩も著しく,現在ではDNAオリガミ専用のソフトウェアも存在する~\cite{douglas2009rapid}.
設計ソフトウェアの存在でDNAオリガミの応用研究は促進され,
機能を持つさまざまなDNAナノ構造体が報告されている.
DNAでつくられたタイルを敷き詰める研究では~\cite{winfree1998design},
論理演算を行ったり~\cite{mao2000logical},
セル・オートマトンのエミュレーションを行うことでDNAタイル演算のチューリング万能性を示す研究がある~\cite{rothemund2004algorithmic}.
これらの研究はDNAナノテクノロジーや情報工学に重要な結果を残しただけでなく,
プログラマブルな物質を使用した自己組織的なパターン形成の制御という観点でも最も成功した例の1つある.
これらの研究例からわかるように,
DNAは先に説明したプログラマブルな物質としての条件を満たしていると考えることができる.
% TODO DNAの説明を詳しく


本研究ではDNAナノテクノロジーの応用の1つのDNAゲルを扱う.
DNAゲルは分岐構造を持つDNA分子を用いてゲルを形成させる技術で~\cite{um2006enzyme},
本研究室の先行研究では,このゲルを用いたマイクロスケールの構造体の作製に成功している.
主に人工細胞の要素技術への応用が考えられており,
その1つは細胞をみたてた脂質二重膜でできたマイクロスケールの小胞の内側にDNAゲルを裏打ちさせる技術である.
DNAゲルの裏打ち構造により,
この小胞は過酷な浸透圧変化に耐えられることが知られている~\cite{kurokawa2017dna}.
また,DNAゲル自身によるカプセル構造の作製も行われている~\cite{morita2017formation}.
遺伝情報をコーディングしたDNAゲルを使用し,
そのゲルの収縮を利用してRAN転写効率を制御する方法が提案されている~\cite{watanabe}.
これらのDNAゲルによるマイクロ構造体はその生体親和性からドラッグデリバリーへの応用も期待される.

このマイクロスケールの構造体の作製においても,自己組織的なパターン形成が応用されている.
1つはゲルを構成するDNA分子形成であるナノスケールのパターン形成で,
もう1つはゲルのマクロな形状を決定するパターン形成である.
前者は熱力学を用いた予測や制御が可能になりつつあるが~\cite{zadeh2011nupack}, 
後者についてはメカニズムが明確に理解されない場合が多い.


\section{研究目的}
本研究では特にDNAゲルカプセルのパターン形成に注目し,
そのメカニズムを主に数値解析により明らかにすることである.
そのために,
DNAゲルカプセルに見られるパターン形成は粘弾性相分離現象というモデルで説明できると仮定し,
カプセルを見立てた球面上に数理モデルを拡張し計算機実験を行った.
計算機実験の結果を先行研究のIn vitroの実験結果と定量的に比較するための解析方法についても開発を行った.
%In vitro の実験を行う上で予測をたてることに

\section{論文構成}
%TODO 後で変更になる
本論文は全6章から構成される.
第1章では,
緒論として自然界のパターン形成のその工学的有用性,
またDNAナノテクノロジーと本研究との関連について述べた.
第2章では,
本研究で扱うDNAゲルカプセルについての詳細を述べる.
この内容は主に本研究室で行われたMoritaらの先行研究に基づく~\cite{morita2017formation}.
また,
パターン形成が脂質に依らないことを裏付けるために行ったIn vitroの補助実験について述べ,
メカニズムが粘弾性相分離現象とよばれるモデルで説明できる可能性について説明する.
第3章では,
計算機実験の手法について述べる.
初めに数理モデルの説明を行い,
私が施した拡張ついて述べる.
次に計算機実験の結果を定量的に解析するために開発した手法について説明する.
第4章では,
計算機実験の結果を述べ,
先行研究の結果との比較を行い考察する.
第5章では,
第2章で述べたIn vitroの実験時に付随して発見したDNAゲルにおける特殊なパターン形成について述べる.
最後に第6章で結論をし総括する.


%----------------
% より詳しいイントロ
%----------------
\chapter{DNAゲルカプセル}

\section{DNAゲル}

本研究で扱うDNAゲルは,Y-DNAと呼ばれる人工的に合成したDNA分子を用いる~\cite{sato2019sequence}.
Y-DNA分子は3本の1本鎖DNAがY字形状にハイブリダイゼーションした構造をとる.
それぞれの1本鎖DNA分子の塩基配列は,
Y字形状をとるように事前にコンピュータソフトウェア~\cite{zadeh2011nupack}を用いて設計される.
3本の腕の先端は完全に相補鎖を形成せず,1本鎖の状態で突出している.
この1本鎖の部分の塩基配列は回文配列になっており,
他の分子の腕の先とハイブリダイゼーションしうる状態になっている.
我々はこの部分を``sticky-ends''と呼んでいる.
%TODO 図を入れる
Y-DNA分子はsticky-endを介して相互作用し,
分子同士のネットワークを形成することでゲルを形成すると考えられている.

\section{DNAゲルカプセル}

DNAゲルをカプセル状に形成したものがDNAゲルカプセルである.



%----------------
% モデルの説明
%----------------
\chapter{実験方法}
% 先行研究に置いて,蛍光観察などからゲルカプセルのパターンはDNAと水分子の2成分から構成されていると考察された.
%TODO 要出典
% この考察からパターン形成は2成分からなる相分離現象により引き起こされると考えた.
% 相分離とは1相からなる混合物が2つ以上の相に分離する現象のことで,
% 例えば油と水の分離や,
% 合金においてみられる現象である.
%TODO 要出典

\section{粘弾性相分離}
液体における通常の相分離では少数成分が液滴を形成し安定状態に落ち着くが,
少数成分を構成する分子が粘弾性力などの相互作用の影響で拡散が遅い場合,
少数成分が液滴を形成する前にネットワークを形成し多数成分が液滴を形成するという通常とは逆の状態が過渡期に観測される.
このような現象は粘弾性相分離と呼ばれ~\cite{tanaka2009formation},
粘性や弾性といった構成分子の動力学的な影響がパターン形成に重要な役割を果たす(図~\ref{fig:veps}).
本研究で注目しているDNAゲルカプセルにおいても,
先の実験結果からY-DNAのsticky-endsによる粘弾性がそのパターン形成に寄与している可能性があり,
粘弾性相分離現象で説明ができるのではないかと考えた.

\begin{figure}
\centering
\includesvg{image/veps.svg}
\caption{
    粘弾性相分離現象の模式図.
    Tanakaらの論文~\cite{tanaka2009formation}より.
}

\label{fig:veps}
\end{figure}

\section{数理モデル}
今回,粘弾性相分離現象の数値解析を行うにあたりArakiらのモデル~\cite{araki2005simple}を用いた.
このモデルでは2次元空間での粘弾性相分離を仮定し,
少数成分を疎視化するラグランジュ型のモデルである.

はじめに粒子(疎視化点)は六方格子状に一様に配置され,
各粒子は隣接する6つの粒子と自然長$0$のバネで接続される.
この状態から,以下のランジュバン方程式を数値的に解いていく.


\begin{figure}
\centering
\includesvg{image/model_2d.svg}
\caption{
    Araki~\cite{araki2005simple}らのモデルの概要.
    (a) 初期状態.計算点の粒子が六方格子上に等間隔に配置され,
        隣接する6つの粒子と自然長$0$で長さ$l_0$のバネにより接続される.
        空間は周期境界条件が適用される.
    (b) 初期配置の拡大図.
    (c) シミュレーションの時間発展の様子.
        粒子の熱ゆらぎと共に,
        接続されていたバネが次第に切断されていくことで粘弾性相分離現象が再現される.
}
\label{fig:model_2d}
\end{figure}


\begin{equation}
\label{eq:main}
\zeta
\frac{d}{dt}
\vec{R}_i
=
-\frac{\partial}{\partial\vec{R}_i}
U_{tot}\{\vec{R}_i\}
+\vec{\xi}_i
.
\end{equation}

$\vec{R}_i$は粒子$i$の位置ベクトル,$\zeta$は粘性抵抗を意味する.
$\vec{\xi}_i$は以下の揺動散逸定理を満たす粒子iにおける熱ゆらぎの項である.

\begin{eqnarray}
\label{eq:langevin0}
\langle\vec{\xi}_i\rangle &=& 0, \\
\label{eq:langevin1}
\langle\vec{\xi}_i(t):\vec{\xi}_j(t')\rangle &=& 2T\zeta\delta_{ij}\delta(t-t')\bm{I}.
\end{eqnarray}

$U_{tot}$は粒子$i$の総ポテンシャルであり,
次のように2つのポテンシャルの和で表される.

\begin{eqnarray}
U_{tot}\{\vec{R}_i\}
&=
 U_{LJ}\{\vec{R}_i\}
+U_{sp}\{\vec{R}_i\}.
\end{eqnarray}

$U_{LJ}$,$U_{sp}$はそれぞれレナード・ジョーンズポテンシャルとバネによる調和ポテンシャルであり,
定義は次のとおりである.

\begin{eqnarray}
U_{LJ}\{\vec{R}_i\}
&=&
4\epsilon\sum_{j\neq i}
\left\{
\left(
\frac{\sigma}{\bigl|\vec{R}_j-\vec{R}_i\bigr|}
\right)^{12}
-
\left(
\frac{\sigma}{\bigl|\vec{R}_j-\vec{R}_i\bigr|}
\right)^{6}
\right\},
\\
U_{sp}\{\vec{R}_i\}
&=&
\frac{1}{2}
\kappa
\sideset{}{'}\sum_{j\neq i}
\bigl|\vec{R}_j-\vec{R}_i\bigr|^{2}.
\end{eqnarray}

$\epsilon$,$\sigma$,$\kappa$はそれぞれポテンシャルを特徴づける定数であり,
調和ポテンシャルにおける記号$\sideset{}{'}\sum$はバネにより接続された粒子間での和を意味する.

初期状態で接続されていたバネは,時間が経つにつれランダムに切断されていく.
その切断確率$p$は温度$T$とバネの長$l$に依存する形で定義され,
各計算ステップで全バネに対して切断するかしないかの処理がなされる.

\begin{eqnarray}
p(l)
&=&
t_0^{-1}
\exp\left(-\frac{\Delta E(l)}{T}\right)
,\\
\Delta E(l)
&=&
E_0-\frac{\kappa l^2}{2}
\end{eqnarray}

具体的には,$0$から$1$の一様乱数$r$を発生させ,$r<p(l)$ならバネを切断する.
$t_0$は数値計算の1ステップの時間で,$E_0$はバネの切れやすさを意味する定数である.
一度切断されたバネは二度と繋がることはなく,粒子間で新たにバネが接続されることもない.


\section{温度項の拡張}
実際のゲルカプセルの作製では温度は一定ではなく,
線形に温度を下げていくアニーリングが施される.
この状況を再現するために,温度項を次のように変更した.

\begin{eqnarray}
T = T_{max}(1-\frac{t}{t_{max}})
\end{eqnarray}

$T_{max}$は初期状態の温度で,$t_{max}$はシミュレーションの終了時間である.
つまり$t=0$で$T=T_{max}$,
$t=t_{max}$で$T=0$になるような線形アニーリングを再現している.


\section{球面上への拡張}
カプセル状での粘弾性相分離現象をシミュレーションするために,
本研究では上記のモデルの球面上への拡張を行った.

まず,ある大きさのGeodesic Grid~\cite{Geodesic}を用意し,
その頂点に粒子を配置し辺の部分をバネとすることで初期状態とする.
この状態からランジュバン方程式を解き,移動した$R_i$を球面上に射影し直すことで常に粒子が球面上にとどまるように計算を繰り返していく.
他の処理に関しては元のモデルと同じである.

%TODO webpage の ref



\section{実装}
本モデルがラグランジュ型のモデルであり,
バネの切断などの特殊な処理が必要であることから,
汎用のシミュレーションソフトウェアでの実験は困難であると考えた.
しかし,モデルは単純であり実装は困難ではないと判断したためシミュレーションプログラムは自ら実装した.

まず,オイラー法による計算を考え,運動方程式(\ref{eq:main})を次のように離散化した.
\begin{eqnarray}
    \vec{R}_i(t+\Delta t) &=& 
    \vec{R}(t)_i
    -\frac{1}{\zeta}\frac{\partial}{\partial\vec{R}_i(t)}U(\vec{R}_i(t))\Delta t
    +\chi\sqrt{\frac{2k_B T}{\zeta}\Delta t}
    .
\end{eqnarray}

次に,各物理量をそれぞれ
$x=l_0 \tilde{x}$,
$t=t_0 \tilde{t}$,
$\zeta=\kappa t_0 \tilde{\zeta}$,
$E=\frac{1}{2}\kappa\l_0^2\tilde{E}$
でスケールし無次元化した.
以下,特に記述がなければ物理量は無次元化されたものとする.


プログラムはC++言語で記述し,ベクトル計算にEigen~\cite{eigenweb}を使用した.
データの可視化にはParaView~\cite{paraview}とPOV-Ray~\cite{povray}をそれぞれ用いた.
計算はラップトップPC上(Intel Core i5-7300U CPU)で並列実行した.


\section{解析}
シミュレーションで得られた構造を定量的に解析する必要がある.
本研究では構造の特徴として孔の大きさの分布を考えた.

得られた構造と,その構造に対して十分小さなグリッドを持つGeodesic Gridを重ね合わせ,
三角形が重なり合わないGeodesic Grid上の連続領域を検出することで孔の検出を行った.

\subsection{計算結果のポリゴン化}
元のモデルでは計算結果は頂点と辺(バネ)のみであるが,
相分離により形成された孔の解析を行うためにポリゴン化を行った.
ポリゴンを構成する三角形は,ある計算点と隣接する計算点で作られる三角形を全て用いた.
また,計算結果をわかりやすくするために配色を変更した.


\begin{figure}
\centering
\includesvg{image/polygon.svg}
\caption{
    計算結果のポリゴン化と配色の変更.
    (a) 変更前の計算結果のCG.
    (b) 同じ計算結果に対する適用後のCG.
}
\label{fig:model_2d}
\end{figure}


\subsection{点と三角形の衝突判定}
孔の検出で行いたい処理は球面上の三角形の衝突判定を繰り返し行うことで実現できる.
そのためにまず,2次元平面のある点が同じ平面上のある三角形の内側に存在するか外側に存在するかを判定するアルゴリズムについて説明する.

平面上の2つのベクトルの外積の値は,時計回りに演算された場合に正に,半時計回りに演算された場合に負になる.
この事実を利用し,

\begin{figure}
\centering
\includesvg{image/point_triangle.svg}
\caption{
    点の三角形内外判定アルゴリズム.
    点$p$から点$q_i$,$q_{(i+1 mod 3)}$へのベクトルの外積を3つの$i$について調べ,
    それらの符号を見ることにより判定できる.
    (a) 点が三角形の外側にある場合は外積の符号が一致しない.
        左2つは外積の値が負になるが,一番右のケースで正になる.
    (b) 点が内側にある場合は外積の符号が全て一致する.
}
\label{fig:point_triangle}
\end{figure}

\subsection{球面上の三角形の衝突判定}


%----------------
% 結果
%----------------
\chapter{結果}
\label{sec:result}

第\ref{sec:model}章で説明した計算機実験の結果とその解析結果について述べる.
平面上のモデルと球面上のモデルについて,
それぞれ温度変化の有無による計算結果を述べ,
最後に先行研究のDNAゲルカプセルとの比較し考察を行う.
%TODO 球面の影響についても言及?


\section{平面モデルの結果}


\subsection{温度変化なし}
まず,今回施した拡張を一切適用しないオリジナルのモデルを用いたシミュレーション結果について述べる.
計算点は縦40,横40の計1600点からなり,長さ$1$のバネで接続された状態から計算を開始する.
各定数については
$\zeta=12000$,$\epsilon=0.4$,$\sigma=0.25$,$\Delta t=1$,$T=0.45, E_0=6.5$の条件で実験を行った.

前半では全体に小さな孔構造が形成されており,
この孔の部分には拡散の速い物質が集合しているとみなすことができる(図\ref{fig:result_2d_without_anearing}b).
時間が経過するにつれ形成された孔が成長し(図\ref{fig:result_2d_without_anearing}c),
系全体でネットワーク構造が形成される(図\ref{fig:result_2d_without_anearing}d).
ネットワーク構造を形成していた粒子は次第にまとまりはじめ(図\ref{fig:result_2d_without_anearing}e-g),
最終的には前半で見られた孔のパターンとは成分が完全に逆転し,
粒子が集合しクラスターを構成する状態で安定となる(図\ref{fig:result_2d_without_anearing}h).
\begin{figure}
    \centering
    \includesvg[scale=0.9]{image/result_2d_without_anearing.svg}
    \caption{
        拡張を施す前の平面モデルの計算結果の時間変化.
        (a) $t=0$.
        (b) $t=10000$.
        (c) $t=20000$.
        (d) $t=30000$.
        (e) $t=40000$.
        (f) $t=50000$.
        (g) $t=60000$.
        (h) $t=70000$.
    }
    \label{fig:result_2d_without_anearing}
\end{figure}
これは粘弾性相分離現象の一連の過程と類似しており,元のモデルが表現していた現象である.


\subsection{温度変化あり}
次に,温度変化の拡張を行った結果について述べる.
温度$T$以外の条件を先の温度変化なしの実験と同じにし,
温度$T$を式\ref{eq:thermal}のように変更し対照実験を行う.
ただし$T_{max}=0.45$,$t_{max}=80000$とし,
この値は先の実験における$T$やシミュレーション時間と同じであるので,
両者の異なる点は$T$の温度が線形に小さくなっていく点のみである.

前半は温度変化のない場合と同じように全体に孔の構造が形成される(図\ref{fig:result_2d_with_anearing}a-d).
\begin{figure}
    \centering
    \includesvg[scale=0.9]{image/result_2d_with_anearing.svg}
    \caption{
        温度変化の拡張を施した平面モデルの計算結果の時間変化.
        (a) $t=0$.
        (b) $t=10000$.
        (c) $t=20000$.
        (d) $t=30000$.
        (e) $t=40000$.
        (f) $t=50000$.
        (g) $t=60000$.
        (h) $t=70000$.
    }
    \label{fig:result_2d_with_anearing}
\end{figure}
後半はその孔が成長するものの,ネットワーク構造を保ったまま安定状態になる(図\ref{fig:result_2d_with_anearing}e-h).
これは温度が下がることにより揺動力やバネの切断確率が小さくなることに起因する.

実際のDNAカプセルゲルも,
通常の粘弾性相分離のように構成成分がクラスターを形成するまで相分離が進行せず,
この実験で得られた結果のように途中ネットワーク構造を形成する段階でゲル化し安定化していると考えられる.


\subsubsection{アニーリング速度依存性}
アニーリング速度を変えた場合のパターンの違いについて述べる.
アニーリング速度を変えて実験するためには,
式\ref{eq:thermal}より$t_{max}$を変更すれば良い.
$t_{max}$が小さいほどアニーリング速度が速いことになる.

アニーリング速度以外の条件は先の実験と同じに設定し,
$t_{max}=20000$,$t_{max}=40000$,$t_{max}=60000$,$t_{max}=80000$の場合について実験を行った(図\ref{fig:result_2d_anearing_speed}).
実験の結果,アニーリング速度により構造に大きな違いが現れることが確認された.
アニーリング速度が大きいほど孔は形成されにくい傾向にあり,
形成されたとしても非常に小さいことが確認できる(図\ref{fig:result_2d_anearing_speed}(a)).
\begin{figure}
    \centering
    \includesvg{image/result_2d_anearing_speed.svg}
    \caption{
        平面モデルにおける構造のアニーリング速度依存性.
        温度変化ありのモデルにおいてアニーリング速度を変えて実験するためには$t_{max}$を変更する.
        式\ref{eq:thermal}より,$t_{max}$が小さいほどアニーリング速度が速い.
        画像は$t=t_{max}$のときの構造のCGである.
        (a) $t_{max}=20000$.
        (b) $t_{max}=40000$.
        (c) $t_{max}=60000$.
        (d) $t_{max}=80000$.
    }
    \label{fig:result_2d_anearing_speed}
\end{figure}
これはバネの切れやすい温度の高い状態にある時間が少ないためであると考えられる.

孔の大きさの分布などの定量的な解析については,
球面モデルの実験結果において議論する.


\section{球面モデルの結果}
次に,モデルを球面上に拡張した場合の結果を述べる.
正二十面体のそれぞれの面を100等分して構築した,1002点からなるGeodesic Gridを初期状態とする.
各定数については平面モデルでの実験と同じである.


\subsection{温度変化なし}
まず,温度変化を行わない条件で実験を行った.
平面モデルの結果と同様に,
孔の成長とクラスターの形成が球面上で確認され
(図\ref{fig:result_sphere_without_anearing}),
\begin{figure}
    \centering
    \includesvg[scale=0.9]{image/result_sphere_without_anearing.svg}
    \caption{
        温度変化の無い球面モデルの計算結果の時間変化.
        (a) $t=0$.
        (b) $t=10000$.
        (c) $t=20000$.
        (d) $t=30000$.
        (e) $t=40000$.
        (f) $t=50000$.
        (g) $t=60000$.
        (h) $t=70000$.
    }
    \label{fig:result_sphere_without_anearing}
\end{figure}
球面上へのモデルの拡張は正しく行われたと考えられる.


\subsection{温度変化あり}
次に,温度変化ありの場合の結果を図\ref{fig:result_sphere_with_anearing}に示す.
\begin{figure}
    \centering
    \includesvg[scale=0.9]{image/result_sphere_with_anearing.svg}
    \caption{
        温度変化の拡張を施した球面モデルの計算結果の時間変化.
        (a) $t=0$.
        (b) $t=10000$.
        (c) $t=20000$.
        (d) $t=30000$.
        (e) $t=40000$.
        (f) $t=50000$.
        (g) $t=60000$.
        (h) $t=70000$.
    }
    \label{fig:result_sphere_with_anearing}
\end{figure}
平面モデルと同じように,
初めに形成された小さな孔が次第に成長する様子が確認された.

実際のDNAゲルカプセルの実験ではアニーリング操作が行われており,
温度変化ありのこの計算機実験が本研究の中心である.
アニーリング速度と粘性抵抗の2種類のパラメータを変更した場合の
最終状態のパターンに関して詳細な解析を行い,
先行研究の結果との比較へ繋げる.


\subsubsection{アニーリング速度依存性}
粘性抵抗$\zeta$の値を$12000$に固定し,
$t_{max}$の値を変えて実験を行った.
$t=t_{max}$の構造の解析を行ったところ,
アニーリング速度が大きいほど($t_{max}$が小さいほど)孔の形成がされにくいことがわかった
(図\ref{fig:result_sphere_anearing_speed_comb}).
\begin{figure}
    \centering
    \includesvg{image/result_sphere_anearing_speed_comb.svg}
    \caption{
        球面モデルにおける構造のアニーリング速度依存性.
        温度変化ありのモデルにおいてアニーリング速度を変えて実験するためには$t_{max}$を変更する.
        式\ref{eq:thermal}より,$t_{max}$が小さいほどアニーリング速度が速い.
        上の画像は,$\zeta=12000$,$t=t_{max}$のときの構造のCGである.
        下のヒストグラムは上の構造の解析結果で,
        横軸は正規化された孔の大きさ,縦軸は頻度(8サンプルの合計個数)である.
        右上の数字は関数$y=ae^{-bx}$で回帰したときの$b$の値で,
        この値が大きいほど孔が形成されにくい.
        アニーリング速度が遅いほど($t_{max}$が大きいほど)
        分布は右側にシフトしており孔が成長しやすいと言える.
        (a) $t_{max}=80000$.
        (b) $t_{max}=60000$.
        (c) $t_{max}=40000$.
        (d) $t_{max}=20000$.
    }
    \label{fig:result_sphere_anearing_speed_comb}
\end{figure}


\subsubsection{粘性抵抗依存性}
次に$t_{max}$の値を$60000$に固定し,
粘性抵抗$\zeta$の値を変えて実験を行った
$t=t_{max}$の構造の解析を行ったところ,
粘性抵抗$\zeta$が大きいほど孔の形成がされにくいことがわかった
(図\ref{fig:result_sphere_friction_constant_comb}).
\begin{figure}
    \centering
    \includesvg{image/result_sphere_friction_constant_comb.svg}
    \caption{
        球面モデルにおける構造の粘性抵抗$\zeta$依存性.
        上の画像は,$t=t_{max}=60000$のときの構造のCGである.
        下のヒストグラムは上の構造の解析結果で,
        横軸は正規化された孔の大きさ,縦軸は頻度(8サンプルの合計個数)である.
        右上の数字は関数$y=ae^{-bx}$で回帰したときの$b$の値で,
        この値が大きいほど孔が形成されにくい.
        $\zeta$が小さいほど分布は右側にシフトしており孔が成長しやすいと言える.
        (a) $\zeta=20000$.
        (b) $\zeta=16000$.
        (c) $\zeta=12000$.
        (d) $\zeta=8000$.
    }
    \label{fig:result_sphere_friction_constant_comb}
\end{figure}


\subsection{両者のまとめ}
以上でアニーリング速度と粘性抵抗の両者の依存性について述べた.
ここで,本研究で用いた全条件の結果を図\ref{fig:result_sphere_all}に示す.
\begin{figure}
    \centering
    \includesvg[scale=0.9]{image/result_sphere_all.svg}
    \caption{
        本研究で用いた全てのアニーリング速度$t_{max}$と
        粘性抵抗$\zeta$の条件対に対する計算機実験の結果のCG.
    }
\label{fig:result_sphere_all}
\end{figure}
また,その孔の分布を図\ref{fig:result_sphere_all_hist}に示す.
\begin{figure}
    \centering
    \includesvg[scale=0.9]{image/result_sphere_all_hist.svg}
    \caption{
        本研究で用いた全てのアニーリング速度$t_{max}$と
        粘性抵抗$\zeta$の条件対に対する計算機実験の結果の解析結果.
        各ヒストグラムの横軸は正規化された孔の大きさ,縦軸は頻度(8サンプルの合計個数)である.
        右上の数字は関数$y=ae^{-bx}$で回帰したときの$b$の値で,
        この値が大きいほど孔が形成されにくい.
    }
    \label{fig:result_sphere_all_hist}
\end{figure}


\section{考察}
本実験では球面上の粘弾性相分離現象による孔構造のできやすさについて,
アニーリング速度と粘性抵抗の2つの観点から結果を得た.
この節では,
これらの結果について実際のDNAゲルカプセルのパターン形成の実験データと対応させながら考察する.

先行研究ではDNAゲルカプセルのパターンを,
孔のあるヘテロ型と一様なホモ型の2つに分けて議論している.
アニーリング速度とカプセルの大きさに対応する
両型の形成割合に関する統計量が実験により得られており(図\ref{fig:result_moritasan}),
\begin{figure}
    \centering
    \includesvg{image/result_moritasan.svg}
    \caption{
        先行研究の結果\cite{moritasan}.
        (a) アニーリング速度依存性.
            アニーリング速度が速いほど孔の構造は形成されにくい.
        (b) 液滴の大きさ依存性.
            液滴の大きさが大きいほど孔の構造は形成されやすい.
    }
    \label{fig:result_moritasan}
\end{figure}
それらのデータと本計算機実験の解析データとを比較した.


\subsection{アニーリング速度依存性}
先行研究の結果によれば,
アニーリング速度が速いほどホモ型のDNAゲルカプセルが形成されやすく(図\ref{fig:result_moritasan}a),
この結果は数値シミュレーションの結果(図\ref{fig:result_sphere_anearing_speed_comb})と一致する.
系の温度がsticky-endsの融解温度を下回るとY-DNA分子は互いに接続し合い,
完全にゲル化してしまう~\cite{sato2019sequence}.
相分離現象が起こるためには物質の流動が必要であるが,
一度ゲル化してしまうと相分離がそれ以上進まないと考えられる.
したがってアニーリング速度が遅いほど,
つまりゲル化するまでの比較的高温の時間が長いほど孔のパターンは形成されやすいと考えられる.

\subsection{カプセルの半径依存性と粘性依存性の関係}
先行研究のもう一方の結果であるカプセルの大きさの依存性に関しては,
直接カプセルの大きさを変更する数値実験を行っていないため,
本モデルを使い次の仮説を検証する形で考察を行う.

第\ref{sec:dnagel}章で述べたとおりDNAゲルカプセルは球状の液滴の内面に形成されるが,
仮に全てのDNA分子が内面に集積しているとすると,
その総量は球の体積に比例するが,
実際に集積するのは球面(面積)でありその厚さは球の半径に比例することがわかる.
したがって,液滴の大きさが異なればDNAのゲル層の厚さも異なると考えられる.
半径$R$の理想的なDNA溶液の液滴を用いてこれを確認する.
溶液の濃度を$\rho$とし,全てのDNA分子が界面に濃度$\rho'$で均一に集積することを考える.
集積したDNAの層の厚さを$d$とすると,集積の前後で分子の総数は保存されるため
\begin{equation}
    \frac{4}{3}\pi R^3 \rho = \left(\frac{4}{3}\pi R^3-\frac{4}{3}\pi(R-d)^3\right)\rho'
\end{equation}
が成りたつ.
これを$d$について解くと
\begin{equation}
    d = R(1-\sqrt[3]{\frac{\rho'-\rho}{\rho'}})
\end{equation}
となり,確かに$d$は$R$に比例することがわかる.

脂質と接するDNA分子は電気的相互作用により脂質膜に吸着するが,
その上に集積してくDNA分子はsticky-endsを介して集合体を形成する.
よって厚く集積するほど脂質膜に吸着しているDNAの割合が小さくなり,
界面上での流動性が高くなる.
以上の議論から,
直径が大きい液滴ほど内面のDNAの流動性は高くなるという仮説をたてた(図~\ref{fig:size_and_friction}).
\begin{figure}
    \centering
    \includesvg{image/size_and_friction.svg}
    \caption{
        DNAゲルカプセル作成時におけるDNA溶液の液滴の大きさと粘性抵抗の関係についての模式図.
        (a) 液滴の直径が大きい場合.
            脂質に近いDNAは電気的な相互作用により引き合うが,
            その上に集積したDNA分子にはほとんど働かず,
            より流動性があると考えられる.
        (b) 液滴の直径が小さい場合.
    }
    \label{fig:size_and_friction}
\end{figure}
ここで言う流動性はモデルにおける粘性抵抗(式\ref{eq:main})と対応すると考えることで,
数値シミュレーション結果における粘性抵抗依存性
(図\ref{fig:result_sphere_friction_constant_comb})
と比較することでこの仮説を肯定的に検証することができる.

先行研究によれば,
液滴の大きさが大きいほどヘテロ型のDNAゲルカプセルが形成されやすく(図\ref{fig:result_moritasan}b), 
この結果は先の仮説を合わせると計算機実験の結果と一致する.


%----------------
% カーンヒリアード
%----------------
\chapter{特殊な相分離現象(仮)}

\section{}


%----------------
% 結論
%----------------
\chapter{結論}

自然界には自己組織化現象に由来する様々なパターン形成現象が存在し,
我々はそれらを数理的に解析することで基礎的な研究に限らず工学的応用も行ってきた.
近年,DNA分子の自己組織化現象を応用したDNAナノテクノロジーが盛んに研究され,
新しいナノ・マイクロスケールのマテリアルやパターン形成制御法として確立されつつある.
DNA分子の利点はそのプログラマビリティにあり,
他の物質では困難な分子構造や物性などを人間が比較的容易に制御できる可能性を持つ.
本研究ではDNAナノテクノロジーの1つであるDNAゲルについて注目し,
DNAゲルが作り出す微小なゲルカプセルにみられるパターン形成メカニズムに関する研究を行った.

DNAゲルカプセルには大きく分けて,孔のようなパターンが形成される場合と,
比較的孔が少なく一様なカプセルが形成される場合の2つが存在する.
先行研究の実験結果と今回補助的に行ったIn Vitroの実験の結果などをふまえ,
カプセルにおける孔の形成が粘弾性相分離と呼ばれる現象に起因すると仮定した(第\ref{sec:dnagel}章).
私は,既存の粘弾性相分離の数理モデルを元に,DNAゲルカプセル用の新しい拡張モデルを構築し,
そのパターン形成の計算機実験を行った.
また,そのパターンについての定量的な議論を行うための解析手法についても開発を行い,
両者のコンピュータプログラムの実装を行った(第\ref{sec:model}章).

計算機実験の結果,
実際のDNAゲルカプセルで見られるパターン形成と非常に類似した現象を再現することに成功した.
また,得られた構造について形成された孔の大きさの分布を見ることで定量的な解析を行った,
温度や粘性抵抗のパラメータを変え,それらの条件依存性について議論した結果,
先行研究で得られていたパターン形成の傾向についての結果を肯定的に説明することができた(第\ref{sec:result}章).

上記の結果は,DNAゲルカプセルに見られるパターンの形成のメカニズムが
粘弾性相分離現象によるものである可能性を結論付け,
さらにそのパターン制御の可能性についても示唆を与えた.

Y-DNA分子はsticky-endsの長さを変更すると,その熱力学的特性の変化から
Y-DNA集合体全体の物性が変化するという結果が報告されている~\cite{sato2019sequence}.
本研究で扱ったDNAカプセルでは,
温度変化やカプセルの大きさ(粘性抵抗)の観点でパターン形成を議論していたが,
sticky-endsの長さによる形成パターンの変化も十分に考えられる.
これは,先に説明したDNA独特の利点を利用したパターン制御であり,
DNAゲルカプセルのパターンがsticky-endsの長さにより制御可能になれば,
その研究は工学的により意義のあるものとなるだろうと考える.
近い将来にそのような研究が行われることは十分に考えられ,
今回作成した数理モデルにsticky-endsの長さに対応する拡張を施すこすことができれば,
そのIn Vitroの実験を計算機実験がサポートしていくのではないかと期待する.

計算機実験の観点でいうと,本研究ではメゾスコピック領域よりも少し大きなスケールを扱った.
ナノスケール領域では分子動力学を用いた計算機実験が盛んに行われており,
マイクロスケール領域ではレオロジーや流体力学などを用いてやはり計算実験が盛んである.
しかし,メゾスコピック領域の計算機実験は十分確立されているとは言えず,
ナノスケールで実験するにはコストがかかりすぎ,
マイクロスケールでは表現が荒すぎるという状況であると考える.
このような観点で言うと,
本研究で扱ったモデルのスケールの領域の計算機実験の研究はさらに行われていくべきと考える.
特にDNAナノテクノロジーの分野においてはY-DNA分子のsticky-endsでわかるように,
連続的な物理定数ではなく塩基配列のような離散的な情報が物性に関与する場合がわかっており,
それらを考慮した新しい形式の数値計算法が生み出される可能性がある.

さらに,上で述べた離散的な情報が物性に関与するという事実は情報工学的にも興味深い.
離散的な塩基配列情報をナチュラルコンピューティングに応用したDNAコンピューティングという分野は存在するが,
それをマテリアルに応用する研究は多くない.
本研究で扱ったDNAゲルはまさにその応用の1つであるが,
その特徴がより強く現われるDNA液滴よ呼ばれるものがある\cite{sato2019sequence}.
DNA液滴は液体のような流動性をもったDNA集合体で,
sticky-endsの配列制御などにより液滴同士の融合分離を制御することができる.
sticky-endsの部分にDNAコンピューティングなどで培われた技術を応用するなどの,
DNAナノテクノロジーと情報工学との別の観点での新しい融合が期待される.


%----------------
% 謝辞
%----------------
\chapter*{謝辞}

% 本研究は,元・瀧ノ上研究室,現・産業技術総合研究所の森田雅宗


%----------------
% 参考文献
%----------------
\bibliographystyle{junsrt}
\bibliography{reference}
% \input{appendix}
% ----------------------------------------------------------------------
\end{document}
