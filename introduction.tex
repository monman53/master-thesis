\chapter{緒論}

% \section{概要}
私達の身の回りの自然界には非常に多種多様なパターンを見出すことができる.
雪の結晶,山や雲の模様,土や塗料のひび割れなどの見慣れたものから,
植物の葉や枝に見られるフラクタル構造や動物の体表の特徴的な模様など,
% TODO ref
生物が自ら作り出すものもある.
パターン形成は目に見えるマクロな世界に限らず,
ミクロな世界でも至るところで行われている.
特に生物の構成要素である細胞では,パターン形成が生命現象に深く関わっていることがわかり始めている.
パターン形成のメカニズムを探ることでその現象の原理や本質を理解できる場合があり,
その行為は基礎科学的にみて重要である.
例えば動物の体表などは反応拡散系というモデルにより良く理解されている.
% TODO ref

さらに自然界のパターン形成の理解は工学的にも有用である.
% 植物の葉脈の例でいうと,そのフラクタル次元を用いた種の同定方法が提案されている.
% TODO ref
人間はその現象を応用し利用してきた. 
% TODO 推敲
自然界にみられるパターン形成は自己組織的に行われることが多く,
この特徴はマテリアルの分野,
特にボトムアップの形成が困難な微小な構造形成に力を発揮する.
スポンジ,特殊表面加工など,例を挙げると枚挙に暇がない.
% TODO もっと例を

先に挙げた例はいずれも重要な技術だが,
化学合成などの従来手法を用いた分子スケールの物質の制御は未だに困難を極める.
したがって,開発には多くの試行錯誤が必要で時間やコストが無視できない.
そこで,人間がプログラマブルなデジタルコンピュータを手に入れたように,
マテリアルの分野においても制御の容易なプログラマブルな物質が求められている.

近年,そのような物質としてDNAが注目されている.
DNAを用いたナノスケールの工学的応用はDNAナノテクノロジーと呼ばれ,
盛んに研究が行われている.
DNAは生物の遺伝情報を担う物質として知られているが,
DNAナノテクノロジーではこれをマテリアルとしても利用する.
% TODO DNAの説明を詳しく
DNAのこの特徴は,先のプログラマブルな物質としての条件を満たす.

DNAナノテクノロジーの応用の1つにDNAゲルがある.
我々はこのゲルを用いたマイクロスケールの構造体の作製に成功している.
主に人工細胞の要素技術への応用が考えられており,
その1つは細胞をみたてた脂質二重膜でできたマイクロスケールの小胞の内側にDNAゲルを裏打ちさせる技術である.
DNAゲルの裏打ち構造により,
この小胞は過酷な浸透圧変化に耐えられることが知られている~\cite{kurokawa2017dna}.
また,DNAゲル自身によるカプセル構造の作製も行われている~\cite{morita2017formation}.
遺伝情報をコーディングしたDNAゲルを使用し,
そのゲルの収縮を利用してRAN転写効率を制御する方法が提案されている~\cite{watanabe}.
これらのDNAゲルによるマイクロ構造体はその生体親和性からドラッグデリバリーへの応用も期待される.

このマイクロスケールの構造体の作製においても,自己組織的なパターン形成が応用されている.
1つはゲルを構成するDNA分子形成であるナノスケールのパターン形成で,
もう1つはゲルのマクロな形状を決定するパターン形成である.
前者は熱力学の応用により予測や制御が可能になりつつあるが~\cite{zadeh2011nupack}, 
後者についてはメカニズムが明確に理解されていない.

そこで,本研究では特にDNAゲルのパターン形成に注目し,
そのメカニズムを主に数値解析により明らかにすることを目的とし,研究を行った.
DNAゲルカプセルに見られるパターン形成は,
粘弾性相分離現象というモデルで説明できると仮定し,
カプセルを見立てた球面上でその数値計算を行った.
その結果と,先行研究で得られていた In vitro の実験結果とを定量的に比較した.
(計算機を用いた実験の利点を記述)
%In vitro の実験を行う上で予測をたてることに
