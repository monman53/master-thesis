\chapter{DNAゲルカプセル}
\label{sec:dnagel}

DNAゲルカプセル作製の実験はMoritaらの先行研究で行われており,
パターン形成に関する実験データが存在する~\cite{morita2017formation}.
私が行った実験はそのパターン形成の解析であるが,
本章ではまずDNAゲルカプセルの作製方法について簡単に説明する.
また,次章移行で扱う数理モデルを決定するために私が行ったIn Vitroの実験についても述べる.

\section{DNAゲルカプセル}

\subsection{DNAゲル}

本研究で扱うDNAゲルは,
Y-DNAと呼ばれる人工的に合成したDNA分子を用いて作製される~\cite{sato2019sequence}.
Y-DNA分子は3本の1本鎖DNAがY字形状にハイブリダイゼーションした構造をとる(図\ref{fig:ydna}b).
3種類ある1本鎖DNA分子のそれぞれの塩基配列は,
Y字形状をとるように事前にコンピュータソフトウェア~\cite{zadeh2011nupack}を用いて設計される.
3本の腕の先端は完全に相補鎖を形成せず,1本鎖の状態で突出している.
この1本鎖の部分の塩基配列は回文配列になっており,
他の分子の腕の先とハイブリダイゼーションしうる状態になっている.
我々はこの部分を``sticky-ends''と呼んでいる.
%TODO 図を入れる
Y-DNA分子同士がsticky-endを介して相互作用しネットワークを形成することで,ゲルが形成される(図\ref{fig:ydna}c).

\begin{figure}
\centering
\includesvg{image/ydna.svg}
\caption{Y-DNAの模式図.
    (a) Y-DNA分子を構成する3種類の一本鎖DNAの塩基配列.
        赤色の部分はsticky-endsと呼ばれる回文配列で,sticky-ends同士は相補の配列である.
    (b) アニーリング操作により3本の一本鎖DNAはY字形状にハイブリダイゼーションする.
    (c) 溶液中でY-DNA分子はsticky-endsを介してネットワークを形成しゲル状態になる.
}

\label{fig:ydna}
\end{figure}

\subsection{DNAゲルカプセル}

DNAゲルをカプセル状に作製したものがDNAゲルカプセルである.
おおまかな作製方法は図~\ref{fig:capsule}の見出しに記述したが,
詳細については次に述べる.

用意するものは液相と油相の2種類の液体である.
油相はミネラルオイルと脂質の混合物で,
脂質はDOPCとDOTAPの二種類から構成される.
% DOPCは帯電していないが,DOTAPは正に帯電している.
液相はY-DNAを構成する1本鎖DNAを水に溶かしたもので,
緩衝剤やナトリウムイオン,DNAの蛍光用色素なども含まれる.

両者を容器に入れ加熱するとY-DNA分子は完全に1本鎖の状態になる.
次に,容器をタッピングし撹拌することでマイクロスケールの液滴を油中に作製する.
液滴と油相の界面には脂質が一重膜を形成しているが,
このときDOTAPは正に帯電しているため膜は正電荷をもつ(DOTAPは帯電していない).
一方でDNA分子は一般に負に帯電しているため,電気的性質により両者は引き合う.
その結果,DNA分子は液滴内面に集積する.
この状態でアニーリング操作を加えることで3種類の1本鎖DNA分子はY-DNA分子を形成し,
しだいにゲル化する.

\begin{figure}
\centering
\includesvg{image/capsule.svg}
\caption{DNAゲルカプセルの作製方法.
    (a) 脂質を混ぜた油とY-DNA溶液を容器に入れ加熱する.
        加熱後,容器を撹拌しマイクロスケールの微小な液滴を作製する.
        この状態でアニーリングを施すことにより,
        液滴内面にDNAゲルが形成されカプセルとなる.
    (b) DNAゲルカプセルが液滴内部で形成される過程の詳細.
        一般に負に帯電しているDNA分子は,正に帯電している脂質膜上に集積する.
        この状態でアニーリング操作を加えることでDNA分子はカプセル状にゲル化する.
        この過程で一様な形状とそうではない形状の2種類のカプセルが形成されることがわかっている.
        この2種類の形成メカニズムを解明することが本研究の主な目的である.
        一番右の画像は実際のDNAゲルカプセルの顕微鏡による蛍光観察の画像で,
        Moritaらの研究結果から拝借した~\cite{moritasan}.
}

\label{fig:capsule}
\end{figure}


\subsection{DNAゲルカプセルにおけるパターン形成}

DNAゲルカプセル作製の過程で形成されるパターンには2種類あることが先行研究により報告されている~\cite{morita2017formation}.
%TODO 推敲
% 実験条件による傾向も確認されているが,
% 両者を分ける明確な理由は明らかになっていない.

ケルカプセルのパターン形成の要因と考えられるものは,
Y-DNA分子,水分子,そして液滴を形成する脂質膜である.
ゲルカプセル作製における脂質膜は2種類の脂質から構成されているが~\cite{morita2017formation},
脂質同士は物性の違いから相分離し脂質膜にパターンが形成されることが知られている~\cite{yanagisawa2014multiple}.

しかし,DNAゲルカプセルのパターン形成と脂質の関係については不明である.
そこで,ゲルカプセルにおけるパターン形成が脂質膜の相分離現象に依るかを確認する実験を行った.

\section{実験}
Y-DNA分子の溶液をそのままアニーリングすることで比較的大きなバルク状態のDNAゲルを作製し,
さらに温度操作を加えることでDNAゲルカプセル作製時と同じような孔構造が形成されるかを確認した.

\subsection{DNA溶液}
DNA溶液は,
Y-DNAの構成要素である3種類の1本鎖DNA,Y1,Y2, Y3それぞれ\SI{5}{\micro M} (ユーロフィンジェノミクス株式会社, 塩基配列は図\ref{fig:ydna}を参照されたい.),
NaCl \SI{200}{mM} (和光純薬工業株式会社),
緩衝剤 Tris-HCl \SI{20}{mM} (ライフテクノロジーズジャパン),
DNA蛍光染色剤 SYBR Gold 1× (ライフテクノロジーズジャパン),
及びMilli-Qから構成され,
1サンプルにつき\SI{20}{\micro L}を作製した.

サンプルはサーマルサイクラーを用いて\SI{85}{\celsius}に加熱し,
\SI{-2}{\celsius/min}のアニーリングを施すことでY-DNAを形成させた.

\subsection{観察}
観察台は,直径\SI{5}{mm}の孔を開けた厚さ4ミリほどのPDMS片をカバーガラス上にプラズマボンディングし,
ガラスの表面に3\%BSA溶液を滴下し15分後に洗い流すことでブロッキングを施すことで作製した.

中央にDNA溶液を全量滴下し,最後に蒸発を防ぐためにミネラルオイルでカバーをした.

この観察台を顕微鏡上のステージヒータに載せ,温度操作を加えながら観察した.
%TODO 顕微鏡・ステージヒータの構成.
\SI{30}{\celsius}の状態から\SI{20}{\celsius/min}で加熱をし\SI{50}{\celsius}で\SI{1}{min}インキュベートする.
そして\SI{-6}{\celsius/min}の冷却をし\SI{30}{\celsius}に戻した.
%TODO 要温度確認


\begin{figure}
\centering
\includesvg{image/equipments.svg}
\caption{Y-DNAの模式図.
    (a) サーマルサイクラー.
    (b) 顕微鏡.
    (c) ステージヒーター.
    (d) 観察台の模式図.
}

\label{fig:equipments}
\end{figure}


\subsection{結果}
観察台にサンプルを滴下し温度操作を加える前はDNAはマイクロスケールのゲル粒子を形成しており,
互いに接着し合い大きなクラスターを形成していた(図\ref{fig:result_dnagel_formation}a).%TODO CCAの話を書く?
加熱を始めると次第にクラスターは収縮し,
一様な大きなバルク状のDNA集合体が形成された(図\ref{fig:result_dnagel_formation}b-c).
インキュベート後に冷却を始めると一様だったDNA集合体に変化が見られ,
小さな孔が無数に形成さた(図\ref{fig:result_dnagel_formation}d).

\begin{figure}
\centering
\includesvg{image/result_dnagel_formation.svg}
\caption{バルクのDNAゲル上の孔形成.
    (a) 加熱前.
    (b) 加熱後.
    (c) インキュベート後.
    (d) 冷却後.
}

\label{fig:result_dnagel_formation}
\end{figure}

冷却が終わるとDNA集合体に変化は見られなくなり,
ゲル状態で安定したと考えられる.

これららの結果から,バルクのDNAゲルにおいても孔のパターンが形成されることが確認された.

\begin{figure}
\centering
\includesvg{image/result_dnagel.svg}
\caption{バルクのDNAゲル上に形成された孔のパターン.
    (a) 孔のパターンを持つDNAゲルの全体像.
    (b) 孔のパターンの拡大画像.
}

\label{fig:result_dnagel}
\end{figure}

\subsection{結論}
脂質膜が存在しない状況でもDNAゲルに孔の形成がみられたことから,
DNAゲルカプセルにみられるパターン形成はY-DNA分子の固有の性質であると考えた.
これは,脂質膜の相分離現象などの従来の現象では説明できない可能性を示唆しており,
通常の相分離モデルとは異なるモデルを考える必要があると考えた.

\section{粘弾性相分離}

先のIn Vitroの実験結果などをふまえ,
本研究ではDNAゲルカプセルのパターン形成が粘弾性相分離と呼ばれる現象に起因するという仮定を置く.
この節では粘弾性相分離現象についての説明と,なぜそのような仮定がなされたかについての説明を行う.
粘弾性相分離の説明の前に,粘性,弾性,粘弾性,相分離,についてそれぞれ辞書的な意味の確認を行う.
これらの定義は岩波理化学辞典第5版~\cite{rikagakuiwanami}より引用した.

\subsection{用語の説明}

\subsubsection{粘性}
運動している流体において,速度勾配のある場合,
速度をならして一様にするような向きの接線応力が現れる性質をいう.

\subsubsection{弾性}
外力によってひずみを受けた物体がそのひずみをもとにもどそうとする力を生ずる性質.

\subsubsection{粘弾性}
緩和現象の結果として,弾性変形と粘性流動が重なって現われる現象.
ふつう液体はずれ変形に対しては粘性を示すだけで弾性は示さないが,
高分子の溶液や融体は粘性とともに弾性をも示す.

\subsubsection{相分離}
熱力学的に均一な1つの状態,すなわち,1つの相にあった物質系が,
温度,圧力などの変数を変化させたとき,
2つの相に分離する現象.

\subsection{DNAゲルカプセルにおける粘弾性相分離現象}
液体における通常の相分離では少数成分が液滴を形成し安定状態に落ち着くが,
少数成分を構成する分子が高分子などの場合,``粘弾性''の定義にあるように弾性力がはたらくことで,
その拡散速度が遅くなる.
この場合,少数成分が液滴を形成する前に少数成分同士でネットワークを形成し多数成分が液滴を形成するという通常とは逆の状態が過渡期に観測されることが知られており(図~\ref{fig:veps}),
このような現象は粘弾性相分離と呼ばれる~\cite{tanaka2009formation},

本研究で扱うDNAゲルカプセルを構成するY-DNA分子は,
1分子あたりの分子量が15万以上であり,さらにそれらがsticky-endsで接続されることを考えると,
十分に高分子としての特徴を備えている.
したがって,Y-DNA分子がゲル形成時に粘弾性相分離現象と同じような現象を引き起こすことが考えられ,
形成メカニズムのモデルとして妥当であると考えた.

\begin{figure}
\centering
\includesvg{image/veps.svg}
\caption{
    粘弾性相分離現象の模式図.
    Tanakaらの論文~\cite{tanaka2009formation}より.
}

\label{fig:veps}
\end{figure}
